%% Generated by Sphinx.
\def\sphinxdocclass{report}
\documentclass[letterpaper,10pt,english]{sphinxmanual}
\ifdefined\pdfpxdimen
   \let\sphinxpxdimen\pdfpxdimen\else\newdimen\sphinxpxdimen
\fi \sphinxpxdimen=.75bp\relax
\ifdefined\pdfimageresolution
    \pdfimageresolution= \numexpr \dimexpr1in\relax/\sphinxpxdimen\relax
\fi
%% let collapsible pdf bookmarks panel have high depth per default
\PassOptionsToPackage{bookmarksdepth=5}{hyperref}

\PassOptionsToPackage{booktabs}{sphinx}
\PassOptionsToPackage{colorrows}{sphinx}

\PassOptionsToPackage{warn}{textcomp}
\usepackage[utf8]{inputenc}
\ifdefined\DeclareUnicodeCharacter
% support both utf8 and utf8x syntaxes
  \ifdefined\DeclareUnicodeCharacterAsOptional
    \def\sphinxDUC#1{\DeclareUnicodeCharacter{"#1}}
  \else
    \let\sphinxDUC\DeclareUnicodeCharacter
  \fi
  \sphinxDUC{00A0}{\nobreakspace}
  \sphinxDUC{2500}{\sphinxunichar{2500}}
  \sphinxDUC{2502}{\sphinxunichar{2502}}
  \sphinxDUC{2514}{\sphinxunichar{2514}}
  \sphinxDUC{251C}{\sphinxunichar{251C}}
  \sphinxDUC{2572}{\textbackslash}
\fi
\usepackage{cmap}
\usepackage[T1]{fontenc}
\usepackage{amsmath,amssymb,amstext}
\usepackage{babel}



\usepackage{tgtermes}
\usepackage{tgheros}
\renewcommand{\ttdefault}{txtt}



\usepackage[Bjarne]{fncychap}
\usepackage{sphinx}

\fvset{fontsize=auto}
\usepackage{geometry}


% Include hyperref last.
\usepackage{hyperref}
% Fix anchor placement for figures with captions.
\usepackage{hypcap}% it must be loaded after hyperref.
% Set up styles of URL: it should be placed after hyperref.
\urlstyle{same}


\usepackage{sphinxmessages}
\setcounter{tocdepth}{1}



\title{Wafer Management}
\date{Aug 04, 2023}
\release{1.0.0}
\author{Romain LAUP}
\newcommand{\sphinxlogo}{\vbox{}}
\renewcommand{\releasename}{Release}
\makeindex
\begin{document}

\ifdefined\shorthandoff
  \ifnum\catcode`\=\string=\active\shorthandoff{=}\fi
  \ifnum\catcode`\"=\active\shorthandoff{"}\fi
\fi

\pagestyle{empty}
\sphinxmaketitle
\pagestyle{plain}
\sphinxtableofcontents
\pagestyle{normal}
\phantomsection\label{\detokenize{index::doc}}


\sphinxstepscope


\chapter{VBD module}
\label{\detokenize{VBD:module-VBD}}\label{\detokenize{VBD:vbd-module}}\label{\detokenize{VBD::doc}}\index{module@\spxentry{module}!VBD@\spxentry{VBD}}\index{VBD@\spxentry{VBD}!module@\spxentry{module}}\index{calculate\_breakdown() (in module VBD)@\spxentry{calculate\_breakdown()}\spxextra{in module VBD}}

\begin{fulllineitems}
\phantomsection\label{\detokenize{VBD:VBD.calculate_breakdown}}
\pysigstartsignatures
\pysiglinewithargsret{\sphinxcode{\sphinxupquote{VBD.}}\sphinxbfcode{\sphinxupquote{calculate\_breakdown}}}{\sphinxparam{\DUrole{n,n}{X}}, \sphinxparam{\DUrole{n,n}{Y}}, \sphinxparam{\DUrole{n,n}{compliance}}}{}
\pysigstopsignatures
\sphinxAtStartPar
This functions calculates Voltage breakdown for two vectors given, and a compliance. Default value of the compliance is set to 1e\sphinxhyphen{}3
\begin{quote}\begin{description}
\sphinxlineitem{Parameters}\begin{itemize}
\item {} 
\sphinxAtStartPar
\sphinxstyleliteralstrong{\sphinxupquote{X}} (\sphinxstyleliteralemphasis{\sphinxupquote{\textless{}list\textgreater{}}}) \textendash{} Values of voltage. Will be converted to a np.array

\item {} 
\sphinxAtStartPar
\sphinxstyleliteralstrong{\sphinxupquote{Y}} (\sphinxstyleliteralemphasis{\sphinxupquote{\textless{}list\textgreater{}}}) \textendash{} Values of current. Will be converted to a np.array and we will take the absolute value of the values

\item {} 
\sphinxAtStartPar
\sphinxstyleliteralstrong{\sphinxupquote{compliance}} (\sphinxstyleliteralemphasis{\sphinxupquote{\textless{}float\textgreater{}}}) \textendash{} Value of compliance. By default is set to 1e\sphinxhyphen{}3

\end{itemize}

\sphinxlineitem{Return \textless{}float\textgreater{} Breakd\_Volt}
\sphinxAtStartPar
Voltage Breakdown. Can be Nan

\sphinxlineitem{Return \textless{}float\textgreater{} Breakd\_Leak}
\sphinxAtStartPar
Leakage Breakdown. Can be Nan

\sphinxlineitem{Return \textless{}bool\textgreater{} reached\_compl}
\sphinxAtStartPar
True if the current reached compliance, False otherwise. If it’s False, Breakd\_Volt will be Nan

\sphinxlineitem{Return \textless{}float\textgreater{} high\_leak}
\sphinxAtStartPar
Compliance if compliance is reached, highest value of the current otherwise

\end{description}\end{quote}

\end{fulllineitems}

\index{create\_wafer\_map() (in module VBD)@\spxentry{create\_wafer\_map()}\spxextra{in module VBD}}

\begin{fulllineitems}
\phantomsection\label{\detokenize{VBD:VBD.create_wafer_map}}
\pysigstartsignatures
\pysiglinewithargsret{\sphinxcode{\sphinxupquote{VBD.}}\sphinxbfcode{\sphinxupquote{create\_wafer\_map}}}{\sphinxparam{\DUrole{n,n}{wafer\_id}}, \sphinxparam{\DUrole{n,n}{session}}, \sphinxparam{\DUrole{n,n}{structure\_id}}}{}
\pysigstopsignatures
\sphinxAtStartPar
This function returns a plot converted to base64, so it can be sent to the User Interface. The plot shows the wafer map based on VBD.
VBD are taken from the database. If a VBD is ‘NaN’, it’s colored in black.
\begin{quote}\begin{description}
\sphinxlineitem{Parameters}\begin{itemize}
\item {} 
\sphinxAtStartPar
\sphinxstyleliteralstrong{\sphinxupquote{wafer\_id}} (\sphinxstyleliteralemphasis{\sphinxupquote{\textless{}str\textgreater{}}}) \textendash{} the name of the wafer

\item {} 
\sphinxAtStartPar
\sphinxstyleliteralstrong{\sphinxupquote{session}} (\sphinxstyleliteralemphasis{\sphinxupquote{\textless{}str\textgreater{}}}) \textendash{} Selected session

\item {} 
\sphinxAtStartPar
\sphinxstyleliteralstrong{\sphinxupquote{structure\_id}} (\sphinxstyleliteralemphasis{\sphinxupquote{\textless{}str\textgreater{}}}) \textendash{} the name of the structure

\end{itemize}

\sphinxlineitem{Return \textless{}list\textgreater{}}
\sphinxAtStartPar
Plot converted to base64

\end{description}\end{quote}

\end{fulllineitems}

\index{fig\_to\_base64() (in module VBD)@\spxentry{fig\_to\_base64()}\spxextra{in module VBD}}

\begin{fulllineitems}
\phantomsection\label{\detokenize{VBD:VBD.fig_to_base64}}
\pysigstartsignatures
\pysiglinewithargsret{\sphinxcode{\sphinxupquote{VBD.}}\sphinxbfcode{\sphinxupquote{fig\_to\_base64}}}{\sphinxparam{\DUrole{n,n}{fig}}}{}
\pysigstopsignatures
\sphinxAtStartPar
Function used to convert a png to base64 to help communication between server and User. Used in ppt\_matrix.
\begin{quote}\begin{description}
\sphinxlineitem{Parameters}
\sphinxAtStartPar
\sphinxstyleliteralstrong{\sphinxupquote{fig}} (\sphinxstyleliteralemphasis{\sphinxupquote{\textless{}png\textgreater{}}}) \textendash{} a figure in png format

\sphinxlineitem{Return \textless{}base64\textgreater{}}
\sphinxAtStartPar
The converted figure

\end{description}\end{quote}

\end{fulllineitems}

\index{get\_all\_x() (in module VBD)@\spxentry{get\_all\_x()}\spxextra{in module VBD}}

\begin{fulllineitems}
\phantomsection\label{\detokenize{VBD:VBD.get_all_x}}
\pysigstartsignatures
\pysiglinewithargsret{\sphinxcode{\sphinxupquote{VBD.}}\sphinxbfcode{\sphinxupquote{get\_all\_x}}}{\sphinxparam{\DUrole{n,n}{wafer\_id}}, \sphinxparam{\DUrole{n,n}{session}}, \sphinxparam{\DUrole{n,n}{structure\_id}}}{}
\pysigstopsignatures
\sphinxAtStartPar
This function gets all coordinates x in a structure. Used to create the wafer map.
\begin{quote}\begin{description}
\sphinxlineitem{Parameters}\begin{itemize}
\item {} 
\sphinxAtStartPar
\sphinxstyleliteralstrong{\sphinxupquote{wafer\_id}} (\sphinxstyleliteralemphasis{\sphinxupquote{\textless{}str\textgreater{}}}) \textendash{} the name of the wafer

\item {} 
\sphinxAtStartPar
\sphinxstyleliteralstrong{\sphinxupquote{session}} (\sphinxstyleliteralemphasis{\sphinxupquote{\textless{}str\textgreater{}}}) \textendash{} name of the session

\item {} 
\sphinxAtStartPar
\sphinxstyleliteralstrong{\sphinxupquote{structure\_id}} (\sphinxstyleliteralemphasis{\sphinxupquote{\textless{}str\textgreater{}}}) \textendash{} the name of the structure

\end{itemize}

\sphinxlineitem{Return \textless{}list\textgreater{}}
\sphinxAtStartPar
List of x in the structure

\end{description}\end{quote}

\end{fulllineitems}

\index{get\_all\_y() (in module VBD)@\spxentry{get\_all\_y()}\spxextra{in module VBD}}

\begin{fulllineitems}
\phantomsection\label{\detokenize{VBD:VBD.get_all_y}}
\pysigstartsignatures
\pysiglinewithargsret{\sphinxcode{\sphinxupquote{VBD.}}\sphinxbfcode{\sphinxupquote{get\_all\_y}}}{\sphinxparam{\DUrole{n,n}{wafer\_id}}, \sphinxparam{\DUrole{n,n}{session}}, \sphinxparam{\DUrole{n,n}{structure\_id}}}{}
\pysigstopsignatures
\sphinxAtStartPar
This function gets all coordinates y in a structure. Used to create the wafer map.
\begin{quote}\begin{description}
\sphinxlineitem{Parameters}\begin{itemize}
\item {} 
\sphinxAtStartPar
\sphinxstyleliteralstrong{\sphinxupquote{wafer\_id}} (\sphinxstyleliteralemphasis{\sphinxupquote{\textless{}str\textgreater{}}}) \textendash{} the name of the wafer

\item {} 
\sphinxAtStartPar
\sphinxstyleliteralstrong{\sphinxupquote{session}} (\sphinxstyleliteralemphasis{\sphinxupquote{\textless{}str\textgreater{}}}) \textendash{} Selected session

\item {} 
\sphinxAtStartPar
\sphinxstyleliteralstrong{\sphinxupquote{structure\_id}} (\sphinxstyleliteralemphasis{\sphinxupquote{\textless{}str\textgreater{}}}) \textendash{} the name of the structure

\end{itemize}

\sphinxlineitem{Return \textless{}list\textgreater{}}
\sphinxAtStartPar
List of y in the structure

\end{description}\end{quote}

\end{fulllineitems}

\index{get\_compliance() (in module VBD)@\spxentry{get\_compliance()}\spxextra{in module VBD}}

\begin{fulllineitems}
\phantomsection\label{\detokenize{VBD:VBD.get_compliance}}
\pysigstartsignatures
\pysiglinewithargsret{\sphinxcode{\sphinxupquote{VBD.}}\sphinxbfcode{\sphinxupquote{get\_compliance}}}{\sphinxparam{\DUrole{n,n}{wafer\_id}}, \sphinxparam{\DUrole{n,n}{session}}}{}
\pysigstopsignatures
\sphinxAtStartPar
This function finds the compliance from the specified session in the database
Returns None if the structure has no compliance registered
\begin{quote}\begin{description}
\sphinxlineitem{Parameters}\begin{itemize}
\item {} 
\sphinxAtStartPar
\sphinxstyleliteralstrong{\sphinxupquote{wafer\_id}} (\sphinxstyleliteralemphasis{\sphinxupquote{\textless{}str\textgreater{}}}) \textendash{} name of the wafer\_id

\item {} 
\sphinxAtStartPar
\sphinxstyleliteralstrong{\sphinxupquote{session}} (\sphinxstyleliteralemphasis{\sphinxupquote{\textless{}str\textgreater{}}}) \textendash{} name of the session

\end{itemize}

\sphinxlineitem{Return \textless{}str\textgreater{}}
\sphinxAtStartPar
the compliance in the wafer

\end{description}\end{quote}

\end{fulllineitems}

\index{get\_vectors\_in\_matrix() (in module VBD)@\spxentry{get\_vectors\_in\_matrix()}\spxextra{in module VBD}}

\begin{fulllineitems}
\phantomsection\label{\detokenize{VBD:VBD.get_vectors_in_matrix}}
\pysigstartsignatures
\pysiglinewithargsret{\sphinxcode{\sphinxupquote{VBD.}}\sphinxbfcode{\sphinxupquote{get\_vectors\_in\_matrix}}}{\sphinxparam{\DUrole{n,n}{wafer\_id}}, \sphinxparam{\DUrole{n,n}{session}}, \sphinxparam{\DUrole{n,n}{structure\_id}}, \sphinxparam{\DUrole{n,n}{x}}, \sphinxparam{\DUrole{n,n}{y}}}{}
\pysigstopsignatures
\sphinxAtStartPar
This function is used to get the values of voltages and current in a matrix. This function is always in parameters for calculate\_breakdown
\begin{quote}\begin{description}
\sphinxlineitem{Parameters}\begin{itemize}
\item {} 
\sphinxAtStartPar
\sphinxstyleliteralstrong{\sphinxupquote{wafer\_id}} (\sphinxstyleliteralemphasis{\sphinxupquote{\textless{}str\textgreater{}}}) \textendash{} the name of the wafer

\item {} 
\sphinxAtStartPar
\sphinxstyleliteralstrong{\sphinxupquote{session}} (\sphinxstyleliteralemphasis{\sphinxupquote{\textless{}str\textgreater{}}}) \textendash{} Selected session

\item {} 
\sphinxAtStartPar
\sphinxstyleliteralstrong{\sphinxupquote{structure\_id}} (\sphinxstyleliteralemphasis{\sphinxupquote{\textless{}str\textgreater{}}}) \textendash{} the name of the structure

\item {} 
\sphinxAtStartPar
\sphinxstyleliteralstrong{\sphinxupquote{x}} (\sphinxstyleliteralemphasis{\sphinxupquote{\textless{}str\textgreater{}}}) \textendash{} the horizontal coordinate of the matrix

\item {} 
\sphinxAtStartPar
\sphinxstyleliteralstrong{\sphinxupquote{y}} (\sphinxstyleliteralemphasis{\sphinxupquote{\textless{}str\textgreater{}}}) \textendash{} the vertical coordinate of the matrix

\end{itemize}

\sphinxlineitem{Return \textless{}list\textgreater{} X}
\sphinxAtStartPar
The values of voltage

\sphinxlineitem{Return \textless{}list\textgreater{} Y}
\sphinxAtStartPar
The values of current

\end{description}\end{quote}

\end{fulllineitems}


\sphinxstepscope


\chapter{WaferMaps module}
\label{\detokenize{WaferMaps:module-WaferMaps}}\label{\detokenize{WaferMaps:wafermaps-module}}\label{\detokenize{WaferMaps::doc}}\index{module@\spxentry{module}!WaferMaps@\spxentry{WaferMaps}}\index{WaferMaps@\spxentry{WaferMaps}!module@\spxentry{module}}\index{C\_wafer\_map() (in module WaferMaps)@\spxentry{C\_wafer\_map()}\spxextra{in module WaferMaps}}

\begin{fulllineitems}
\phantomsection\label{\detokenize{WaferMaps:WaferMaps.C_wafer_map}}
\pysigstartsignatures
\pysiglinewithargsret{\sphinxcode{\sphinxupquote{WaferMaps.}}\sphinxbfcode{\sphinxupquote{C\_wafer\_map}}}{\sphinxparam{\DUrole{n,n}{wafer\_id}}, \sphinxparam{\DUrole{n,n}{session}}, \sphinxparam{\DUrole{n,n}{structure\_id}}}{}
\pysigstopsignatures
\sphinxAtStartPar
This function returns a plot converted to base64, so it can be sent to the User Interface. The plot shows the wafer map based on C.
C are taken from the database.
\begin{quote}\begin{description}
\sphinxlineitem{Parameters}\begin{itemize}
\item {} 
\sphinxAtStartPar
\sphinxstyleliteralstrong{\sphinxupquote{wafer\_id}} (\sphinxstyleliteralemphasis{\sphinxupquote{\textless{}str\textgreater{}}}) \textendash{} the name of the wafer

\item {} 
\sphinxAtStartPar
\sphinxstyleliteralstrong{\sphinxupquote{session}} (\sphinxstyleliteralemphasis{\sphinxupquote{\textless{}str\textgreater{}}}) \textendash{} Selected session

\item {} 
\sphinxAtStartPar
\sphinxstyleliteralstrong{\sphinxupquote{structure\_id}} (\sphinxstyleliteralemphasis{\sphinxupquote{\textless{}str\textgreater{}}}) \textendash{} the name of the structure

\end{itemize}

\sphinxlineitem{Return \textless{}list\textgreater{}}
\sphinxAtStartPar
Plot converted to base64

\end{description}\end{quote}

\end{fulllineitems}

\index{Cmes\_wafer\_map() (in module WaferMaps)@\spxentry{Cmes\_wafer\_map()}\spxextra{in module WaferMaps}}

\begin{fulllineitems}
\phantomsection\label{\detokenize{WaferMaps:WaferMaps.Cmes_wafer_map}}
\pysigstartsignatures
\pysiglinewithargsret{\sphinxcode{\sphinxupquote{WaferMaps.}}\sphinxbfcode{\sphinxupquote{Cmes\_wafer\_map}}}{\sphinxparam{\DUrole{n,n}{wafer\_id}}, \sphinxparam{\DUrole{n,n}{session}}, \sphinxparam{\DUrole{n,n}{structure\_id}}}{}
\pysigstopsignatures
\sphinxAtStartPar
This function returns a plot converted to base64, so it can be sent to the User Interface. The plot shows the wafer map based on Cmes.
Cmes are taken from the database.
\begin{quote}\begin{description}
\sphinxlineitem{Parameters}\begin{itemize}
\item {} 
\sphinxAtStartPar
\sphinxstyleliteralstrong{\sphinxupquote{wafer\_id}} (\sphinxstyleliteralemphasis{\sphinxupquote{\textless{}str\textgreater{}}}) \textendash{} the name of the wafer

\item {} 
\sphinxAtStartPar
\sphinxstyleliteralstrong{\sphinxupquote{session}} (\sphinxstyleliteralemphasis{\sphinxupquote{\textless{}str\textgreater{}}}) \textendash{} Selected session

\item {} 
\sphinxAtStartPar
\sphinxstyleliteralstrong{\sphinxupquote{structure\_id}} (\sphinxstyleliteralemphasis{\sphinxupquote{\textless{}str\textgreater{}}}) \textendash{} the name of the structure

\end{itemize}

\sphinxlineitem{Return \textless{}list\textgreater{}}
\sphinxAtStartPar
Plot converted to base64

\end{description}\end{quote}

\end{fulllineitems}

\index{Leak\_wafer\_map() (in module WaferMaps)@\spxentry{Leak\_wafer\_map()}\spxextra{in module WaferMaps}}

\begin{fulllineitems}
\phantomsection\label{\detokenize{WaferMaps:WaferMaps.Leak_wafer_map}}
\pysigstartsignatures
\pysiglinewithargsret{\sphinxcode{\sphinxupquote{WaferMaps.}}\sphinxbfcode{\sphinxupquote{Leak\_wafer\_map}}}{\sphinxparam{\DUrole{n,n}{wafer\_id}}, \sphinxparam{\DUrole{n,n}{session}}, \sphinxparam{\DUrole{n,n}{structure\_id}}}{}
\pysigstopsignatures
\sphinxAtStartPar
This function returns a plot converted to base64, so it can be sent to the User Interface. The plot shows the wafer map based on Leak.
Leak are taken from the database.
\begin{quote}\begin{description}
\sphinxlineitem{Parameters}\begin{itemize}
\item {} 
\sphinxAtStartPar
\sphinxstyleliteralstrong{\sphinxupquote{wafer\_id}} (\sphinxstyleliteralemphasis{\sphinxupquote{\textless{}str\textgreater{}}}) \textendash{} the name of the wafer

\item {} 
\sphinxAtStartPar
\sphinxstyleliteralstrong{\sphinxupquote{session}} (\sphinxstyleliteralemphasis{\sphinxupquote{\textless{}str\textgreater{}}}) \textendash{} Selected session

\item {} 
\sphinxAtStartPar
\sphinxstyleliteralstrong{\sphinxupquote{structure\_id}} (\sphinxstyleliteralemphasis{\sphinxupquote{\textless{}str\textgreater{}}}) \textendash{} the name of the structure

\end{itemize}

\sphinxlineitem{Return \textless{}list\textgreater{}}
\sphinxAtStartPar
Plot converted to base64

\end{description}\end{quote}

\end{fulllineitems}

\index{R\_wafer\_map() (in module WaferMaps)@\spxentry{R\_wafer\_map()}\spxextra{in module WaferMaps}}

\begin{fulllineitems}
\phantomsection\label{\detokenize{WaferMaps:WaferMaps.R_wafer_map}}
\pysigstartsignatures
\pysiglinewithargsret{\sphinxcode{\sphinxupquote{WaferMaps.}}\sphinxbfcode{\sphinxupquote{R\_wafer\_map}}}{\sphinxparam{\DUrole{n,n}{wafer\_id}}, \sphinxparam{\DUrole{n,n}{session}}, \sphinxparam{\DUrole{n,n}{structure\_id}}}{}
\pysigstopsignatures
\sphinxAtStartPar
This function returns a plot converted to base64, so it can be sent to the User Interface. The plot shows the wafer map based on R.
R are taken from the database. If an R is a broken value, like 999999 or 999997, it’s colored in black.
\begin{quote}\begin{description}
\sphinxlineitem{Parameters}\begin{itemize}
\item {} 
\sphinxAtStartPar
\sphinxstyleliteralstrong{\sphinxupquote{wafer\_id}} (\sphinxstyleliteralemphasis{\sphinxupquote{\textless{}str\textgreater{}}}) \textendash{} the name of the wafer

\item {} 
\sphinxAtStartPar
\sphinxstyleliteralstrong{\sphinxupquote{session}} (\sphinxstyleliteralemphasis{\sphinxupquote{\textless{}str\textgreater{}}}) \textendash{} Selected session

\item {} 
\sphinxAtStartPar
\sphinxstyleliteralstrong{\sphinxupquote{structure\_id}} (\sphinxstyleliteralemphasis{\sphinxupquote{\textless{}str\textgreater{}}}) \textendash{} the name of the structure

\end{itemize}

\sphinxlineitem{Return \textless{}list\textgreater{}}
\sphinxAtStartPar
Plot converted to base64

\end{description}\end{quote}

\end{fulllineitems}


\sphinxstepscope


\chapter{app module}
\label{\detokenize{app:module-app}}\label{\detokenize{app:app-module}}\label{\detokenize{app::doc}}\index{module@\spxentry{module}!app@\spxentry{app}}\index{app@\spxentry{app}!module@\spxentry{module}}\index{C\_normal() (in module app)@\spxentry{C\_normal()}\spxextra{in module app}}

\begin{fulllineitems}
\phantomsection\label{\detokenize{app:app.C_normal}}
\pysigstartsignatures
\pysiglinewithargsret{\sphinxcode{\sphinxupquote{app.}}\sphinxbfcode{\sphinxupquote{C\_normal}}}{\sphinxparam{\DUrole{n,n}{waferId}}, \sphinxparam{\DUrole{n,n}{sessions}}, \sphinxparam{\DUrole{n,n}{structures}}, \sphinxparam{\DUrole{n,n}{coords}}}{}
\pysigstopsignatures
\sphinxAtStartPar
Used for plotting the normal plots of C.
We first refactor received information and then call the function

\end{fulllineitems}

\index{Cmes\_normal() (in module app)@\spxentry{Cmes\_normal()}\spxextra{in module app}}

\begin{fulllineitems}
\phantomsection\label{\detokenize{app:app.Cmes_normal}}
\pysigstartsignatures
\pysiglinewithargsret{\sphinxcode{\sphinxupquote{app.}}\sphinxbfcode{\sphinxupquote{Cmes\_normal}}}{\sphinxparam{\DUrole{n,n}{waferId}}, \sphinxparam{\DUrole{n,n}{sessions}}, \sphinxparam{\DUrole{n,n}{structures}}, \sphinxparam{\DUrole{n,n}{coords}}}{}
\pysigstopsignatures
\sphinxAtStartPar
Used for plotting the normal plots of Cmes.
We first refactor received information and then call the function

\end{fulllineitems}

\index{Leak\_normal() (in module app)@\spxentry{Leak\_normal()}\spxextra{in module app}}

\begin{fulllineitems}
\phantomsection\label{\detokenize{app:app.Leak_normal}}
\pysigstartsignatures
\pysiglinewithargsret{\sphinxcode{\sphinxupquote{app.}}\sphinxbfcode{\sphinxupquote{Leak\_normal}}}{\sphinxparam{\DUrole{n,n}{waferId}}, \sphinxparam{\DUrole{n,n}{sessions}}, \sphinxparam{\DUrole{n,n}{structures}}, \sphinxparam{\DUrole{n,n}{coords}}}{}
\pysigstopsignatures
\sphinxAtStartPar
Used for plotting the normal plots of Leak.
We first refactor received information and then call the function

\end{fulllineitems}

\index{R\_normal() (in module app)@\spxentry{R\_normal()}\spxextra{in module app}}

\begin{fulllineitems}
\phantomsection\label{\detokenize{app:app.R_normal}}
\pysigstartsignatures
\pysiglinewithargsret{\sphinxcode{\sphinxupquote{app.}}\sphinxbfcode{\sphinxupquote{R\_normal}}}{\sphinxparam{\DUrole{n,n}{waferId}}, \sphinxparam{\DUrole{n,n}{sessions}}, \sphinxparam{\DUrole{n,n}{structures}}, \sphinxparam{\DUrole{n,n}{coords}}}{}
\pysigstopsignatures
\sphinxAtStartPar
Used for plotting the normal plots of R.
We first refactor received information and then call the function

\end{fulllineitems}

\index{VBD\_normal() (in module app)@\spxentry{VBD\_normal()}\spxextra{in module app}}

\begin{fulllineitems}
\phantomsection\label{\detokenize{app:app.VBD_normal}}
\pysigstartsignatures
\pysiglinewithargsret{\sphinxcode{\sphinxupquote{app.}}\sphinxbfcode{\sphinxupquote{VBD\_normal}}}{\sphinxparam{\DUrole{n,n}{waferId}}, \sphinxparam{\DUrole{n,n}{sessions}}, \sphinxparam{\DUrole{n,n}{structures}}, \sphinxparam{\DUrole{n,n}{coords}}}{}
\pysigstopsignatures
\sphinxAtStartPar
Used for plotting the normal plots of selected VBD.
We first refactor received information and then call the function

\end{fulllineitems}

\index{delete\_wafer() (in module app)@\spxentry{delete\_wafer()}\spxextra{in module app}}

\begin{fulllineitems}
\phantomsection\label{\detokenize{app:app.delete_wafer}}
\pysigstartsignatures
\pysiglinewithargsret{\sphinxcode{\sphinxupquote{app.}}\sphinxbfcode{\sphinxupquote{delete\_wafer}}}{\sphinxparam{\DUrole{n,n}{wafer\_id}}}{}
\pysigstopsignatures
\sphinxAtStartPar
Used for deleting a wafer

\end{fulllineitems}

\index{excel\_structure\_route() (in module app)@\spxentry{excel\_structure\_route()}\spxextra{in module app}}

\begin{fulllineitems}
\phantomsection\label{\detokenize{app:app.excel_structure_route}}
\pysigstartsignatures
\pysiglinewithargsret{\sphinxcode{\sphinxupquote{app.}}\sphinxbfcode{\sphinxupquote{excel\_structure\_route}}}{\sphinxparam{\DUrole{n,n}{waferId}}, \sphinxparam{\DUrole{n,n}{sessions}}, \sphinxparam{\DUrole{n,n}{structures}}, \sphinxparam{\DUrole{n,n}{types}}, \sphinxparam{\DUrole{n,n}{temps}}, \sphinxparam{\DUrole{n,n}{files}}, \sphinxparam{\DUrole{n,n}{coords}}, \sphinxparam{\DUrole{n,n}{file\_name}}}{}
\pysigstopsignatures
\sphinxAtStartPar
Used for creating an Excel with selected parameters. We first refactor received information and then call the function

\end{fulllineitems}

\index{filter\_by\_Coords() (in module app)@\spxentry{filter\_by\_Coords()}\spxextra{in module app}}

\begin{fulllineitems}
\phantomsection\label{\detokenize{app:app.filter_by_Coords}}
\pysigstartsignatures
\pysiglinewithargsret{\sphinxcode{\sphinxupquote{app.}}\sphinxbfcode{\sphinxupquote{filter\_by\_Coords}}}{\sphinxparam{\DUrole{n,n}{wafer\_id}}, \sphinxparam{\DUrole{n,n}{selectedMeasurement}}}{}
\pysigstopsignatures
\sphinxAtStartPar
Used for displaying all structures that contain the selected coordinates inside the given wafer.

\end{fulllineitems}

\index{filter\_by\_Filenames() (in module app)@\spxentry{filter\_by\_Filenames()}\spxextra{in module app}}

\begin{fulllineitems}
\phantomsection\label{\detokenize{app:app.filter_by_Filenames}}
\pysigstartsignatures
\pysiglinewithargsret{\sphinxcode{\sphinxupquote{app.}}\sphinxbfcode{\sphinxupquote{filter\_by\_Filenames}}}{\sphinxparam{\DUrole{n,n}{wafer\_id}}, \sphinxparam{\DUrole{n,n}{selectedMeasurement}}}{}
\pysigstopsignatures
\sphinxAtStartPar
Used for displaying all structures that contain the selected filename inside the given wafer.

\end{fulllineitems}

\index{filter\_by\_Meas() (in module app)@\spxentry{filter\_by\_Meas()}\spxextra{in module app}}

\begin{fulllineitems}
\phantomsection\label{\detokenize{app:app.filter_by_Meas}}
\pysigstartsignatures
\pysiglinewithargsret{\sphinxcode{\sphinxupquote{app.}}\sphinxbfcode{\sphinxupquote{filter\_by\_Meas}}}{\sphinxparam{\DUrole{n,n}{wafer\_id}}, \sphinxparam{\DUrole{n,n}{selectedMeasurement}}}{}
\pysigstopsignatures
\sphinxAtStartPar
Used for displaying all structures that contain the selected type of measure inside the given wafer.

\end{fulllineitems}

\index{filter\_by\_Session() (in module app)@\spxentry{filter\_by\_Session()}\spxextra{in module app}}

\begin{fulllineitems}
\phantomsection\label{\detokenize{app:app.filter_by_Session}}
\pysigstartsignatures
\pysiglinewithargsret{\sphinxcode{\sphinxupquote{app.}}\sphinxbfcode{\sphinxupquote{filter\_by\_Session}}}{\sphinxparam{\DUrole{n,n}{wafer\_id}}, \sphinxparam{\DUrole{n,n}{selectedMeasurement}}}{}
\pysigstopsignatures
\sphinxAtStartPar
Used for displaying all structures that contain the selected session inside the given wafer.

\end{fulllineitems}

\index{filter\_by\_Temps() (in module app)@\spxentry{filter\_by\_Temps()}\spxextra{in module app}}

\begin{fulllineitems}
\phantomsection\label{\detokenize{app:app.filter_by_Temps}}
\pysigstartsignatures
\pysiglinewithargsret{\sphinxcode{\sphinxupquote{app.}}\sphinxbfcode{\sphinxupquote{filter\_by\_Temps}}}{\sphinxparam{\DUrole{n,n}{wafer\_id}}, \sphinxparam{\DUrole{n,n}{selectedMeasurement}}}{}
\pysigstopsignatures
\sphinxAtStartPar
Used for displaying all structures that contain the selected temperature inside the given wafer.

\end{fulllineitems}

\index{get\_all\_coords() (in module app)@\spxentry{get\_all\_coords()}\spxextra{in module app}}

\begin{fulllineitems}
\phantomsection\label{\detokenize{app:app.get_all_coords}}
\pysigstartsignatures
\pysiglinewithargsret{\sphinxcode{\sphinxupquote{app.}}\sphinxbfcode{\sphinxupquote{get\_all\_coords}}}{\sphinxparam{\DUrole{n,n}{wafer\_id}}}{}
\pysigstopsignatures
\sphinxAtStartPar
Used for getting all coordinates inside a given wafer.

\end{fulllineitems}

\index{get\_all\_filenames() (in module app)@\spxentry{get\_all\_filenames()}\spxextra{in module app}}

\begin{fulllineitems}
\phantomsection\label{\detokenize{app:app.get_all_filenames}}
\pysigstartsignatures
\pysiglinewithargsret{\sphinxcode{\sphinxupquote{app.}}\sphinxbfcode{\sphinxupquote{get\_all\_filenames}}}{\sphinxparam{\DUrole{n,n}{wafer\_id}}}{}
\pysigstopsignatures
\sphinxAtStartPar
Used for getting all filenames inside a given wafer.

\end{fulllineitems}

\index{get\_all\_structures() (in module app)@\spxentry{get\_all\_structures()}\spxextra{in module app}}

\begin{fulllineitems}
\phantomsection\label{\detokenize{app:app.get_all_structures}}
\pysigstartsignatures
\pysiglinewithargsret{\sphinxcode{\sphinxupquote{app.}}\sphinxbfcode{\sphinxupquote{get\_all\_structures}}}{\sphinxparam{\DUrole{n,n}{wafer\_id}}}{}
\pysigstopsignatures
\sphinxAtStartPar
Used for getting all structures in all sessions in a wafer.

\end{fulllineitems}

\index{get\_all\_temps() (in module app)@\spxentry{get\_all\_temps()}\spxextra{in module app}}

\begin{fulllineitems}
\phantomsection\label{\detokenize{app:app.get_all_temps}}
\pysigstartsignatures
\pysiglinewithargsret{\sphinxcode{\sphinxupquote{app.}}\sphinxbfcode{\sphinxupquote{get\_all\_temps}}}{\sphinxparam{\DUrole{n,n}{wafer\_id}}}{}
\pysigstopsignatures
\sphinxAtStartPar
Used for getting all temperatures inside a given wafer.

\end{fulllineitems}

\index{get\_all\_types() (in module app)@\spxentry{get\_all\_types()}\spxextra{in module app}}

\begin{fulllineitems}
\phantomsection\label{\detokenize{app:app.get_all_types}}
\pysigstartsignatures
\pysiglinewithargsret{\sphinxcode{\sphinxupquote{app.}}\sphinxbfcode{\sphinxupquote{get\_all\_types}}}{\sphinxparam{\DUrole{n,n}{wafer\_id}}}{}
\pysigstopsignatures
\sphinxAtStartPar
Used for getting all types of measures inside a given wafer.

\end{fulllineitems}

\index{get\_compl() (in module app)@\spxentry{get\_compl()}\spxextra{in module app}}

\begin{fulllineitems}
\phantomsection\label{\detokenize{app:app.get_compl}}
\pysigstartsignatures
\pysiglinewithargsret{\sphinxcode{\sphinxupquote{app.}}\sphinxbfcode{\sphinxupquote{get\_compl}}}{\sphinxparam{\DUrole{n,n}{waferId}}, \sphinxparam{\DUrole{n,n}{session}}}{}
\pysigstopsignatures
\sphinxAtStartPar
Used for getting the compliance of the selected session

\end{fulllineitems}

\index{get\_map\_sessions\_server() (in module app)@\spxentry{get\_map\_sessions\_server()}\spxextra{in module app}}

\begin{fulllineitems}
\phantomsection\label{\detokenize{app:app.get_map_sessions_server}}
\pysigstartsignatures
\pysiglinewithargsret{\sphinxcode{\sphinxupquote{app.}}\sphinxbfcode{\sphinxupquote{get\_map\_sessions\_server}}}{\sphinxparam{\DUrole{n,n}{wafer\_id}}}{}
\pysigstopsignatures
\sphinxAtStartPar
Used for getting all sessions that contain I\sphinxhyphen{}V measurements (for the wafer map)

\end{fulllineitems}

\index{get\_map\_structures\_server() (in module app)@\spxentry{get\_map\_structures\_server()}\spxextra{in module app}}

\begin{fulllineitems}
\phantomsection\label{\detokenize{app:app.get_map_structures_server}}
\pysigstartsignatures
\pysiglinewithargsret{\sphinxcode{\sphinxupquote{app.}}\sphinxbfcode{\sphinxupquote{get\_map\_structures\_server}}}{\sphinxparam{\DUrole{n,n}{wafer\_id}}, \sphinxparam{\DUrole{n,n}{session}}}{}
\pysigstopsignatures
\sphinxAtStartPar
Used for getting all structures that contain I\sphinxhyphen{}V measurements (for the wafer map)

\end{fulllineitems}

\index{get\_matrices() (in module app)@\spxentry{get\_matrices()}\spxextra{in module app}}

\begin{fulllineitems}
\phantomsection\label{\detokenize{app:app.get_matrices}}
\pysigstartsignatures
\pysiglinewithargsret{\sphinxcode{\sphinxupquote{app.}}\sphinxbfcode{\sphinxupquote{get\_matrices}}}{\sphinxparam{\DUrole{n,n}{wafer\_id}}, \sphinxparam{\DUrole{n,n}{structure\_id}}}{}
\pysigstopsignatures
\sphinxAtStartPar
Used for getting all dies in the structure selected

\end{fulllineitems}

\index{get\_normal\_values() (in module app)@\spxentry{get\_normal\_values()}\spxextra{in module app}}

\begin{fulllineitems}
\phantomsection\label{\detokenize{app:app.get_normal_values}}
\pysigstartsignatures
\pysiglinewithargsret{\sphinxcode{\sphinxupquote{app.}}\sphinxbfcode{\sphinxupquote{get\_normal\_values}}}{\sphinxparam{\DUrole{n,n}{waferId}}}{}
\pysigstopsignatures
\sphinxAtStartPar
Used for getting all extracted values in a wafer (Leak, R, C, Cmes and/or VBD)

\end{fulllineitems}

\index{get\_sessions\_server() (in module app)@\spxentry{get\_sessions\_server()}\spxextra{in module app}}

\begin{fulllineitems}
\phantomsection\label{\detokenize{app:app.get_sessions_server}}
\pysigstartsignatures
\pysiglinewithargsret{\sphinxcode{\sphinxupquote{app.}}\sphinxbfcode{\sphinxupquote{get\_sessions\_server}}}{\sphinxparam{\DUrole{n,n}{wafer\_id}}}{}
\pysigstopsignatures
\sphinxAtStartPar
Used for getting all sessions inside a given wafer.

\end{fulllineitems}

\index{get\_structures\_json() (in module app)@\spxentry{get\_structures\_json()}\spxextra{in module app}}

\begin{fulllineitems}
\phantomsection\label{\detokenize{app:app.get_structures_json}}
\pysigstartsignatures
\pysiglinewithargsret{\sphinxcode{\sphinxupquote{app.}}\sphinxbfcode{\sphinxupquote{get\_structures\_json}}}{\sphinxparam{\DUrole{n,n}{wafer\_id}}, \sphinxparam{\DUrole{n,n}{session}}}{}
\pysigstopsignatures
\sphinxAtStartPar
Used for getting all structures inside a given session in a wafer.

\end{fulllineitems}

\index{index() (in module app)@\spxentry{index()}\spxextra{in module app}}

\begin{fulllineitems}
\phantomsection\label{\detokenize{app:app.index}}
\pysigstartsignatures
\pysiglinewithargsret{\sphinxcode{\sphinxupquote{app.}}\sphinxbfcode{\sphinxupquote{index}}}{}{}
\pysigstopsignatures
\sphinxAtStartPar
Used for set up the app

\end{fulllineitems}

\index{open() (in module app)@\spxentry{open()}\spxextra{in module app}}

\begin{fulllineitems}
\phantomsection\label{\detokenize{app:app.open}}
\pysigstartsignatures
\pysiglinewithargsret{\sphinxcode{\sphinxupquote{app.}}\sphinxbfcode{\sphinxupquote{open}}}{}{}
\pysigstopsignatures
\sphinxAtStartPar
Used for displaying all registered wafers.

\end{fulllineitems}

\index{options() (in module app)@\spxentry{options()}\spxextra{in module app}}

\begin{fulllineitems}
\phantomsection\label{\detokenize{app:app.options}}
\pysigstartsignatures
\pysiglinewithargsret{\sphinxcode{\sphinxupquote{app.}}\sphinxbfcode{\sphinxupquote{options}}}{\sphinxparam{\DUrole{n,n}{checkbox\_checked}}}{}
\pysigstopsignatures
\sphinxAtStartPar
Used for registering data in the database. We check if the user wants to register J\sphinxhyphen{}V measures.
Then, we use the correct function for each type of file (txt, tbl or lim)
tbl files are converted into txt.
Finally, Upload Folder is cleared

\end{fulllineitems}

\index{personal\_wafer\_map() (in module app)@\spxentry{personal\_wafer\_map()}\spxextra{in module app}}

\begin{fulllineitems}
\phantomsection\label{\detokenize{app:app.personal_wafer_map}}
\pysigstartsignatures
\pysiglinewithargsret{\sphinxcode{\sphinxupquote{app.}}\sphinxbfcode{\sphinxupquote{personal\_wafer\_map}}}{\sphinxparam{\DUrole{n,n}{waferId}}, \sphinxparam{\DUrole{n,n}{session}}, \sphinxparam{\DUrole{n,n}{structure}}}{}
\pysigstopsignatures
\sphinxAtStartPar
Used for plotting the wafer map of the selected structure

\end{fulllineitems}

\index{plot\_we\_want() (in module app)@\spxentry{plot\_we\_want()}\spxextra{in module app}}

\begin{fulllineitems}
\phantomsection\label{\detokenize{app:app.plot_we_want}}
\pysigstartsignatures
\pysiglinewithargsret{\sphinxcode{\sphinxupquote{app.}}\sphinxbfcode{\sphinxupquote{plot\_we\_want}}}{\sphinxparam{\DUrole{n,n}{waferId}}, \sphinxparam{\DUrole{n,n}{sessions}}, \sphinxparam{\DUrole{n,n}{structures}}, \sphinxparam{\DUrole{n,n}{types}}, \sphinxparam{\DUrole{n,n}{temps}}, \sphinxparam{\DUrole{n,n}{files}}, \sphinxparam{\DUrole{n,n}{coords}}}{}
\pysigstopsignatures
\sphinxAtStartPar
Used for plotting the dies selected with selected parameters. We first refactor received information and then call the function

\end{fulllineitems}

\index{ppt\_structure\_route() (in module app)@\spxentry{ppt\_structure\_route()}\spxextra{in module app}}

\begin{fulllineitems}
\phantomsection\label{\detokenize{app:app.ppt_structure_route}}
\pysigstartsignatures
\pysiglinewithargsret{\sphinxcode{\sphinxupquote{app.}}\sphinxbfcode{\sphinxupquote{ppt\_structure\_route}}}{\sphinxparam{\DUrole{n,n}{waferId}}, \sphinxparam{\DUrole{n,n}{sessions}}, \sphinxparam{\DUrole{n,n}{structures}}, \sphinxparam{\DUrole{n,n}{types}}, \sphinxparam{\DUrole{n,n}{temps}}, \sphinxparam{\DUrole{n,n}{files}}, \sphinxparam{\DUrole{n,n}{coords}}, \sphinxparam{\DUrole{n,n}{file\_name}}}{}
\pysigstopsignatures
\sphinxAtStartPar
Used for creating a Powerpoint with selected parameters. We first refactor received information and then call the function

\end{fulllineitems}

\index{reg\_excel\_VBD() (in module app)@\spxentry{reg\_excel\_VBD()}\spxextra{in module app}}

\begin{fulllineitems}
\phantomsection\label{\detokenize{app:app.reg_excel_VBD}}
\pysigstartsignatures
\pysiglinewithargsret{\sphinxcode{\sphinxupquote{app.}}\sphinxbfcode{\sphinxupquote{reg\_excel\_VBD}}}{\sphinxparam{\DUrole{n,n}{waferId}}, \sphinxparam{\DUrole{n,n}{sessions}}, \sphinxparam{\DUrole{n,n}{structures}}, \sphinxparam{\DUrole{n,n}{temps}}, \sphinxparam{\DUrole{n,n}{files}}, \sphinxparam{\DUrole{n,n}{coords}}, \sphinxparam{\DUrole{n,n}{file\_name}}}{}
\pysigstopsignatures
\sphinxAtStartPar
Used for saving selected VBDs in an excel file.
We first refactor received information and then call the function

\end{fulllineitems}

\index{send\_css() (in module app)@\spxentry{send\_css()}\spxextra{in module app}}

\begin{fulllineitems}
\phantomsection\label{\detokenize{app:app.send_css}}
\pysigstartsignatures
\pysiglinewithargsret{\sphinxcode{\sphinxupquote{app.}}\sphinxbfcode{\sphinxupquote{send\_css}}}{\sphinxparam{\DUrole{n,n}{path}}}{}
\pysigstopsignatures
\sphinxAtStartPar
Used for set up the app

\end{fulllineitems}

\index{send\_js() (in module app)@\spxentry{send\_js()}\spxextra{in module app}}

\begin{fulllineitems}
\phantomsection\label{\detokenize{app:app.send_js}}
\pysigstartsignatures
\pysiglinewithargsret{\sphinxcode{\sphinxupquote{app.}}\sphinxbfcode{\sphinxupquote{send\_js}}}{\sphinxparam{\DUrole{n,n}{path}}}{}
\pysigstopsignatures
\sphinxAtStartPar
Used for set up the app

\end{fulllineitems}

\index{send\_static\_files() (in module app)@\spxentry{send\_static\_files()}\spxextra{in module app}}

\begin{fulllineitems}
\phantomsection\label{\detokenize{app:app.send_static_files}}
\pysigstartsignatures
\pysiglinewithargsret{\sphinxcode{\sphinxupquote{app.}}\sphinxbfcode{\sphinxupquote{send\_static\_files}}}{\sphinxparam{\DUrole{n,n}{path}}}{}
\pysigstopsignatures
\sphinxAtStartPar
Used for set up the app

\end{fulllineitems}

\index{serve() (in module app)@\spxentry{serve()}\spxextra{in module app}}

\begin{fulllineitems}
\phantomsection\label{\detokenize{app:app.serve}}
\pysigstartsignatures
\pysiglinewithargsret{\sphinxcode{\sphinxupquote{app.}}\sphinxbfcode{\sphinxupquote{serve}}}{\sphinxparam{\DUrole{n,n}{path}}}{}
\pysigstopsignatures
\sphinxAtStartPar
Used for set up the app

\end{fulllineitems}

\index{set\_compl() (in module app)@\spxentry{set\_compl()}\spxextra{in module app}}

\begin{fulllineitems}
\phantomsection\label{\detokenize{app:app.set_compl}}
\pysigstartsignatures
\pysiglinewithargsret{\sphinxcode{\sphinxupquote{app.}}\sphinxbfcode{\sphinxupquote{set\_compl}}}{\sphinxparam{\DUrole{n,n}{waferId}}, \sphinxparam{\DUrole{n,n}{session}}, \sphinxparam{\DUrole{n,n}{compliance}}}{}
\pysigstopsignatures
\sphinxAtStartPar
Used for setting the compliance of the selected session

\end{fulllineitems}

\index{upload() (in module app)@\spxentry{upload()}\spxextra{in module app}}

\begin{fulllineitems}
\phantomsection\label{\detokenize{app:app.upload}}
\pysigstartsignatures
\pysiglinewithargsret{\sphinxcode{\sphinxupquote{app.}}\sphinxbfcode{\sphinxupquote{upload}}}{}{}
\pysigstopsignatures
\sphinxAtStartPar
Used for collect files dropped by user. We create an Upload Folder and then handle each file:
Step 1: uncompress file if it is a compressed file
Step 2: Manage lim files and file without extension

\end{fulllineitems}


\sphinxstepscope


\chapter{converter module}
\label{\detokenize{converter:module-converter}}\label{\detokenize{converter:converter-module}}\label{\detokenize{converter::doc}}\index{module@\spxentry{module}!converter@\spxentry{converter}}\index{converter@\spxentry{converter}!module@\spxentry{module}}\index{handle\_file() (in module converter)@\spxentry{handle\_file()}\spxextra{in module converter}}

\begin{fulllineitems}
\phantomsection\label{\detokenize{converter:converter.handle_file}}
\pysigstartsignatures
\pysiglinewithargsret{\sphinxcode{\sphinxupquote{converter.}}\sphinxbfcode{\sphinxupquote{handle\_file}}}{\sphinxparam{\DUrole{n,n}{file\_path}}}{}
\pysigstopsignatures
\sphinxAtStartPar
This function takes the path of a file and uncompress and convert it into a txt file.
Files that can be handled: .tbl.Z, .tbl and .txt
After conversion, the original file is deleted from the DataFiles folder
\begin{quote}\begin{description}
\sphinxlineitem{Parameters}
\sphinxAtStartPar
\sphinxstyleliteralstrong{\sphinxupquote{file\_path}} (\sphinxstyleliteralemphasis{\sphinxupquote{\textless{}str\textgreater{}}}) \textendash{} path of the file to be converted

\sphinxlineitem{Return \textless{}str\textgreater{}}
\sphinxAtStartPar
path of the converted file

\end{description}\end{quote}

\end{fulllineitems}

\index{tbl\_to\_txt() (in module converter)@\spxentry{tbl\_to\_txt()}\spxextra{in module converter}}

\begin{fulllineitems}
\phantomsection\label{\detokenize{converter:converter.tbl_to_txt}}
\pysigstartsignatures
\pysiglinewithargsret{\sphinxcode{\sphinxupquote{converter.}}\sphinxbfcode{\sphinxupquote{tbl\_to\_txt}}}{\sphinxparam{\DUrole{n,n}{path}}}{}
\pysigstopsignatures
\sphinxAtStartPar
Converts a tbl file into a txt file. The file is read, all information is converted and saved in a list and then we write all at once in an empty txt file.
\begin{quote}\begin{description}
\sphinxlineitem{Parameters}
\sphinxAtStartPar
\sphinxstyleliteralstrong{\sphinxupquote{path}} \textendash{} Path of the file to be converted

\end{description}\end{quote}

\end{fulllineitems}

\index{traducer() (in module converter)@\spxentry{traducer()}\spxextra{in module converter}}

\begin{fulllineitems}
\phantomsection\label{\detokenize{converter:converter.traducer}}
\pysigstartsignatures
\pysiglinewithargsret{\sphinxcode{\sphinxupquote{converter.}}\sphinxbfcode{\sphinxupquote{traducer}}}{\sphinxparam{\DUrole{n,n}{line}}}{}
\pysigstopsignatures
\sphinxAtStartPar
Used for converting tbl files into txt files. Parse a line and returns information into the format used in txt files.
This function is called in tbl\_to\_txt.
\begin{quote}\begin{description}
\sphinxlineitem{Parameters}
\sphinxAtStartPar
\sphinxstyleliteralstrong{\sphinxupquote{line}} (\sphinxstyleliteralemphasis{\sphinxupquote{\textless{}str\textgreater{}}}) \textendash{} Line to be parsed

\sphinxlineitem{Returns}
\sphinxAtStartPar
Line converted to the format used in the txt files

\end{description}\end{quote}

\end{fulllineitems}


\sphinxstepscope


\chapter{excel module}
\label{\detokenize{excel:module-excel}}\label{\detokenize{excel:excel-module}}\label{\detokenize{excel::doc}}\index{module@\spxentry{module}!excel@\spxentry{excel}}\index{excel@\spxentry{excel}!module@\spxentry{module}}\index{classify\_row() (in module excel)@\spxentry{classify\_row()}\spxextra{in module excel}}

\begin{fulllineitems}
\phantomsection\label{\detokenize{excel:excel.classify_row}}
\pysigstartsignatures
\pysiglinewithargsret{\sphinxcode{\sphinxupquote{excel.}}\sphinxbfcode{\sphinxupquote{classify\_row}}}{\sphinxparam{\DUrole{n,n}{row}}}{}
\pysigstopsignatures
\sphinxAtStartPar
Used to know in which sheet the corresponding row has to go
\begin{quote}\begin{description}
\sphinxlineitem{Parameters}
\sphinxAtStartPar
\sphinxstyleliteralstrong{\sphinxupquote{row}} \textendash{} Row of the DataFrame

\sphinxlineitem{Returns}
\sphinxAtStartPar
A str : ‘Positives’, ‘Negatives’ or ‘NaN’

\end{description}\end{quote}

\end{fulllineitems}

\index{excel\_VBD() (in module excel)@\spxentry{excel\_VBD()}\spxextra{in module excel}}

\begin{fulllineitems}
\phantomsection\label{\detokenize{excel:excel.excel_VBD}}
\pysigstartsignatures
\pysiglinewithargsret{\sphinxcode{\sphinxupquote{excel.}}\sphinxbfcode{\sphinxupquote{excel\_VBD}}}{\sphinxparam{\DUrole{n,n}{wafer\_id}}, \sphinxparam{\DUrole{n,n}{sessions}}, \sphinxparam{\DUrole{n,n}{structures}}, \sphinxparam{\DUrole{n,n}{Temps}}, \sphinxparam{\DUrole{n,n}{Files}}, \sphinxparam{\DUrole{n,n}{coords}}, \sphinxparam{\DUrole{n,n}{file\_name}}}{}
\pysigstopsignatures
\sphinxAtStartPar
An Excel file is created with all VBDs inside the wafer, following selected filters.
3 sheets are created: one for positive values, one for negatives and one for NaN
\begin{quote}\begin{description}
\sphinxlineitem{Parameters}\begin{itemize}
\item {} 
\sphinxAtStartPar
\sphinxstyleliteralstrong{\sphinxupquote{wafer\_id}} \textendash{} ID of the wafer

\item {} 
\sphinxAtStartPar
\sphinxstyleliteralstrong{\sphinxupquote{sessions}} \textendash{} Filtered sessions

\item {} 
\sphinxAtStartPar
\sphinxstyleliteralstrong{\sphinxupquote{structures}} \textendash{} Filtered structures

\item {} 
\sphinxAtStartPar
\sphinxstyleliteralstrong{\sphinxupquote{Temps}} \textendash{} Filtered temperatures

\item {} 
\sphinxAtStartPar
\sphinxstyleliteralstrong{\sphinxupquote{Files}} \textendash{} Filtered files

\item {} 
\sphinxAtStartPar
\sphinxstyleliteralstrong{\sphinxupquote{coords}} \textendash{} Filtered coordinates

\item {} 
\sphinxAtStartPar
\sphinxstyleliteralstrong{\sphinxupquote{file\_name}} \textendash{} Name under which the file will be registered

\end{itemize}

\end{description}\end{quote}

\end{fulllineitems}

\index{wanted\_excel() (in module excel)@\spxentry{wanted\_excel()}\spxextra{in module excel}}

\begin{fulllineitems}
\phantomsection\label{\detokenize{excel:excel.wanted_excel}}
\pysigstartsignatures
\pysiglinewithargsret{\sphinxcode{\sphinxupquote{excel.}}\sphinxbfcode{\sphinxupquote{wanted\_excel}}}{\sphinxparam{\DUrole{n,n}{wafer\_id}}, \sphinxparam{\DUrole{n,n}{sessions}}, \sphinxparam{\DUrole{n,n}{structures}}, \sphinxparam{\DUrole{n,n}{types}}, \sphinxparam{\DUrole{n,n}{Temps}}, \sphinxparam{\DUrole{n,n}{Files}}, \sphinxparam{\DUrole{n,n}{coords}}, \sphinxparam{\DUrole{n,n}{file\_name}}}{}
\pysigstopsignatures
\sphinxAtStartPar
This function creates an Excel file with given information and register it in a folder named following the concerned wafer.
We first get all information from the database into a Pandas’ DataFrame, and then we write the DataFrame into an Excel.
One column is: {[}name of the session, name of the structure, Unit of the Measure, {[}Measures{]}{]}
One sheet is created per die and one file is created per type of measure
Size of columns are automatically adjusted for better readability.
\begin{quote}\begin{description}
\sphinxlineitem{Parameters}\begin{itemize}
\item {} 
\sphinxAtStartPar
\sphinxstyleliteralstrong{\sphinxupquote{wafer\_id}} \textendash{} ID of the Wafer

\item {} 
\sphinxAtStartPar
\sphinxstyleliteralstrong{\sphinxupquote{sessions}} \textendash{} All sessions that we want to write

\item {} 
\sphinxAtStartPar
\sphinxstyleliteralstrong{\sphinxupquote{structures}} \textendash{} All structures that we want to write

\item {} 
\sphinxAtStartPar
\sphinxstyleliteralstrong{\sphinxupquote{types}} \textendash{} All types that we want to write

\item {} 
\sphinxAtStartPar
\sphinxstyleliteralstrong{\sphinxupquote{Temps}} \textendash{} All temperatures that we want to write

\item {} 
\sphinxAtStartPar
\sphinxstyleliteralstrong{\sphinxupquote{Files}} \textendash{} All files that we want to write

\item {} 
\sphinxAtStartPar
\sphinxstyleliteralstrong{\sphinxupquote{coords}} \textendash{} All coordinates that we want to write

\item {} 
\sphinxAtStartPar
\sphinxstyleliteralstrong{\sphinxupquote{file\_name}} \textendash{} Name under which the file will be registered

\end{itemize}

\end{description}\end{quote}

\end{fulllineitems}


\sphinxstepscope


\chapter{filter module}
\label{\detokenize{filter:module-filter}}\label{\detokenize{filter:filter-module}}\label{\detokenize{filter::doc}}\index{module@\spxentry{module}!filter@\spxentry{filter}}\index{filter@\spxentry{filter}!module@\spxentry{module}}\index{filter\_by\_coord() (in module filter)@\spxentry{filter\_by\_coord()}\spxextra{in module filter}}

\begin{fulllineitems}
\phantomsection\label{\detokenize{filter:filter.filter_by_coord}}
\pysigstartsignatures
\pysiglinewithargsret{\sphinxcode{\sphinxupquote{filter.}}\sphinxbfcode{\sphinxupquote{filter\_by\_coord}}}{\sphinxparam{\DUrole{n,n}{coords}}, \sphinxparam{\DUrole{n,n}{wafer\_id}}}{}
\pysigstopsignatures
\sphinxAtStartPar
This function browse the database to find all structures that have a matrix with the couple of coordinates specified.
\begin{quote}\begin{description}
\sphinxlineitem{Parameters}\begin{itemize}
\item {} 
\sphinxAtStartPar
\sphinxstyleliteralstrong{\sphinxupquote{coords}} (\sphinxstyleliteralemphasis{\sphinxupquote{\textless{}list\textgreater{}}}) \textendash{} List of str, contains all couples of coordinates that want to be matched

\item {} 
\sphinxAtStartPar
\sphinxstyleliteralstrong{\sphinxupquote{wafer\_id}} (\sphinxstyleliteralemphasis{\sphinxupquote{\textless{}str\textgreater{}}}) \textendash{} Name of the wafer\_id

\end{itemize}

\sphinxlineitem{Return \textless{}list\textgreater{}}
\sphinxAtStartPar
A list of structures that contains the specified coordinates

\end{description}\end{quote}

\end{fulllineitems}

\index{filter\_by\_filename() (in module filter)@\spxentry{filter\_by\_filename()}\spxextra{in module filter}}

\begin{fulllineitems}
\phantomsection\label{\detokenize{filter:filter.filter_by_filename}}
\pysigstartsignatures
\pysiglinewithargsret{\sphinxcode{\sphinxupquote{filter.}}\sphinxbfcode{\sphinxupquote{filter\_by\_filename}}}{\sphinxparam{\DUrole{n,n}{file=\textless{}class \textquotesingle{}str\textquotesingle{}\textgreater{}}}, \sphinxparam{\DUrole{n,n}{wafer\_id=\textless{}class \textquotesingle{}str\textquotesingle{}\textgreater{}}}}{}
\pysigstopsignatures
\sphinxAtStartPar
This function browse the database to find all structures that have measurements from thr file specified.
\begin{quote}\begin{description}
\sphinxlineitem{Parameters}\begin{itemize}
\item {} 
\sphinxAtStartPar
\sphinxstyleliteralstrong{\sphinxupquote{file}} (\sphinxstyleliteralemphasis{\sphinxupquote{\textless{}list\textgreater{}}}) \textendash{} List of str, contains all files that wants to be matched

\item {} 
\sphinxAtStartPar
\sphinxstyleliteralstrong{\sphinxupquote{wafer\_id}} (\sphinxstyleliteralemphasis{\sphinxupquote{\textless{}str\textgreater{}}}) \textendash{} Name of the wafer\_id

\end{itemize}

\sphinxlineitem{Return \textless{}list\textgreater{}}
\sphinxAtStartPar
A list of structures that contains the specified files

\end{description}\end{quote}

\end{fulllineitems}

\index{filter\_by\_meas() (in module filter)@\spxentry{filter\_by\_meas()}\spxextra{in module filter}}

\begin{fulllineitems}
\phantomsection\label{\detokenize{filter:filter.filter_by_meas}}
\pysigstartsignatures
\pysiglinewithargsret{\sphinxcode{\sphinxupquote{filter.}}\sphinxbfcode{\sphinxupquote{filter\_by\_meas}}}{\sphinxparam{\DUrole{n,n}{meas}}, \sphinxparam{\DUrole{n,n}{wafer\_id}}}{}
\pysigstopsignatures
\sphinxAtStartPar
This function browse the database to find all structures that have the types of measurements specified.
\begin{quote}\begin{description}
\sphinxlineitem{Parameters}\begin{itemize}
\item {} 
\sphinxAtStartPar
\sphinxstyleliteralstrong{\sphinxupquote{meas}} (\sphinxstyleliteralemphasis{\sphinxupquote{\textless{}list\textgreater{}}}) \textendash{} List of str, contains all Measurements that wants to be matched

\item {} 
\sphinxAtStartPar
\sphinxstyleliteralstrong{\sphinxupquote{wafer\_id}} (\sphinxstyleliteralemphasis{\sphinxupquote{\textless{}str\textgreater{}}}) \textendash{} Name of the wafer\_id

\end{itemize}

\sphinxlineitem{Return \textless{}list\textgreater{}}
\sphinxAtStartPar
A list of structures that contains the specified measurements

\end{description}\end{quote}

\end{fulllineitems}

\index{filter\_by\_session() (in module filter)@\spxentry{filter\_by\_session()}\spxextra{in module filter}}

\begin{fulllineitems}
\phantomsection\label{\detokenize{filter:filter.filter_by_session}}
\pysigstartsignatures
\pysiglinewithargsret{\sphinxcode{\sphinxupquote{filter.}}\sphinxbfcode{\sphinxupquote{filter\_by\_session}}}{\sphinxparam{\DUrole{n,n}{session=\textless{}class \textquotesingle{}str\textquotesingle{}\textgreater{}}}, \sphinxparam{\DUrole{n,n}{wafer\_id=\textless{}class \textquotesingle{}str\textquotesingle{}\textgreater{}}}}{}
\pysigstopsignatures
\sphinxAtStartPar
This function browse the database to find all structures that have measurements from thr file specified.
\begin{quote}\begin{description}
\sphinxlineitem{Parameters}\begin{itemize}
\item {} 
\sphinxAtStartPar
\sphinxstyleliteralstrong{\sphinxupquote{session}} (\sphinxstyleliteralemphasis{\sphinxupquote{\textless{}list\textgreater{}}}) \textendash{} List of str, contains all files that wants to be matched

\item {} 
\sphinxAtStartPar
\sphinxstyleliteralstrong{\sphinxupquote{wafer\_id}} (\sphinxstyleliteralemphasis{\sphinxupquote{\textless{}str\textgreater{}}}) \textendash{} Name of the wafer\_id

\end{itemize}

\sphinxlineitem{Return \textless{}list\textgreater{}}
\sphinxAtStartPar
A list of structures that contains the specified session

\end{description}\end{quote}

\end{fulllineitems}

\index{filter\_by\_temp() (in module filter)@\spxentry{filter\_by\_temp()}\spxextra{in module filter}}

\begin{fulllineitems}
\phantomsection\label{\detokenize{filter:filter.filter_by_temp}}
\pysigstartsignatures
\pysiglinewithargsret{\sphinxcode{\sphinxupquote{filter.}}\sphinxbfcode{\sphinxupquote{filter\_by\_temp}}}{\sphinxparam{\DUrole{n,n}{temps}}, \sphinxparam{\DUrole{n,n}{wafer\_id}}}{}
\pysigstopsignatures
\sphinxAtStartPar
This function browse the database to find all structures that have the temperature specified.
\begin{quote}\begin{description}
\sphinxlineitem{Parameters}\begin{itemize}
\item {} 
\sphinxAtStartPar
\sphinxstyleliteralstrong{\sphinxupquote{temps}} (\sphinxstyleliteralemphasis{\sphinxupquote{\textless{}list\textgreater{}}}) \textendash{} List of str, contains all temperatures that wants to be matched

\item {} 
\sphinxAtStartPar
\sphinxstyleliteralstrong{\sphinxupquote{wafer\_id}} (\sphinxstyleliteralemphasis{\sphinxupquote{\textless{}str\textgreater{}}}) \textendash{} Name of the wafer\_id

\end{itemize}

\sphinxlineitem{Return \textless{}list\textgreater{}}
\sphinxAtStartPar
A list of structures that contains the specified temperatures

\end{description}\end{quote}

\end{fulllineitems}


\sphinxstepscope


\chapter{getter module}
\label{\detokenize{getter:module-getter}}\label{\detokenize{getter:getter-module}}\label{\detokenize{getter::doc}}\index{module@\spxentry{module}!getter@\spxentry{getter}}\index{getter@\spxentry{getter}!module@\spxentry{module}}\index{connexion() (in module getter)@\spxentry{connexion()}\spxextra{in module getter}}

\begin{fulllineitems}
\phantomsection\label{\detokenize{getter:getter.connexion}}
\pysigstartsignatures
\pysiglinewithargsret{\sphinxcode{\sphinxupquote{getter.}}\sphinxbfcode{\sphinxupquote{connexion}}}{}{}
\pysigstopsignatures
\sphinxAtStartPar
Used to connect to the database, so we can manipulate data. /! Only use to read data, not to write data in the DB /!
:return: The collection.

\end{fulllineitems}

\index{get\_C\_sessions() (in module getter)@\spxentry{get\_C\_sessions()}\spxextra{in module getter}}

\begin{fulllineitems}
\phantomsection\label{\detokenize{getter:getter.get_C_sessions}}
\pysigstartsignatures
\pysiglinewithargsret{\sphinxcode{\sphinxupquote{getter.}}\sphinxbfcode{\sphinxupquote{get\_C\_sessions}}}{\sphinxparam{\DUrole{n,n}{wafer\_id}}}{}
\pysigstopsignatures
\sphinxAtStartPar
Used to get all sessions that contain C values (for wafer maps)
\begin{quote}\begin{description}
\sphinxlineitem{Parameters}
\sphinxAtStartPar
\sphinxstyleliteralstrong{\sphinxupquote{wafer\_id}} (\sphinxstyleliteralemphasis{\sphinxupquote{\textless{}str\textgreater{}}}) \textendash{} ID of the wafer

\sphinxlineitem{Return \textless{}list\textgreater{}}
\sphinxAtStartPar
List of all sessions that contain C values

\end{description}\end{quote}

\end{fulllineitems}

\index{get\_C\_structures() (in module getter)@\spxentry{get\_C\_structures()}\spxextra{in module getter}}

\begin{fulllineitems}
\phantomsection\label{\detokenize{getter:getter.get_C_structures}}
\pysigstartsignatures
\pysiglinewithargsret{\sphinxcode{\sphinxupquote{getter.}}\sphinxbfcode{\sphinxupquote{get\_C\_structures}}}{\sphinxparam{\DUrole{n,n}{wafer\_id}}, \sphinxparam{\DUrole{n,n}{session}}}{}
\pysigstopsignatures
\sphinxAtStartPar
Used to get all structures that contain C values (for wafer maps)
\begin{quote}\begin{description}
\sphinxlineitem{Parameters}\begin{itemize}
\item {} 
\sphinxAtStartPar
\sphinxstyleliteralstrong{\sphinxupquote{wafer\_id}} (\sphinxstyleliteralemphasis{\sphinxupquote{\textless{}str\textgreater{}}}) \textendash{} ID of the wafer

\item {} 
\sphinxAtStartPar
\sphinxstyleliteralstrong{\sphinxupquote{session}} (\sphinxstyleliteralemphasis{\sphinxupquote{\textless{}str\textgreater{}}}) \textendash{} Name of the session

\end{itemize}

\sphinxlineitem{Return \textless{}list\textgreater{}}
\sphinxAtStartPar
List of all structures that contain C values

\end{description}\end{quote}

\end{fulllineitems}

\index{get\_Cmes\_sessions() (in module getter)@\spxentry{get\_Cmes\_sessions()}\spxextra{in module getter}}

\begin{fulllineitems}
\phantomsection\label{\detokenize{getter:getter.get_Cmes_sessions}}
\pysigstartsignatures
\pysiglinewithargsret{\sphinxcode{\sphinxupquote{getter.}}\sphinxbfcode{\sphinxupquote{get\_Cmes\_sessions}}}{\sphinxparam{\DUrole{n,n}{wafer\_id}}}{}
\pysigstopsignatures
\sphinxAtStartPar
Used to get all sessions that contain Cmes values (for wafer maps)
\begin{quote}\begin{description}
\sphinxlineitem{Parameters}
\sphinxAtStartPar
\sphinxstyleliteralstrong{\sphinxupquote{wafer\_id}} (\sphinxstyleliteralemphasis{\sphinxupquote{\textless{}str\textgreater{}}}) \textendash{} ID of the wafer

\sphinxlineitem{Return \textless{}list\textgreater{}}
\sphinxAtStartPar
List of all sessions that contain Cmes values

\end{description}\end{quote}

\end{fulllineitems}

\index{get\_Cmes\_structures() (in module getter)@\spxentry{get\_Cmes\_structures()}\spxextra{in module getter}}

\begin{fulllineitems}
\phantomsection\label{\detokenize{getter:getter.get_Cmes_structures}}
\pysigstartsignatures
\pysiglinewithargsret{\sphinxcode{\sphinxupquote{getter.}}\sphinxbfcode{\sphinxupquote{get\_Cmes\_structures}}}{\sphinxparam{\DUrole{n,n}{wafer\_id}}, \sphinxparam{\DUrole{n,n}{session}}}{}
\pysigstopsignatures
\sphinxAtStartPar
Used to get all structures that contain Cmes values (for wafer maps)
\begin{quote}\begin{description}
\sphinxlineitem{Parameters}\begin{itemize}
\item {} 
\sphinxAtStartPar
\sphinxstyleliteralstrong{\sphinxupquote{wafer\_id}} (\sphinxstyleliteralemphasis{\sphinxupquote{\textless{}str\textgreater{}}}) \textendash{} ID of the wafer

\item {} 
\sphinxAtStartPar
\sphinxstyleliteralstrong{\sphinxupquote{session}} (\sphinxstyleliteralemphasis{\sphinxupquote{\textless{}str\textgreater{}}}) \textendash{} Name of the session

\end{itemize}

\sphinxlineitem{Return \textless{}list\textgreater{}}
\sphinxAtStartPar
List of all structures that contain Cmes values

\end{description}\end{quote}

\end{fulllineitems}

\index{get\_Leak\_sessions() (in module getter)@\spxentry{get\_Leak\_sessions()}\spxextra{in module getter}}

\begin{fulllineitems}
\phantomsection\label{\detokenize{getter:getter.get_Leak_sessions}}
\pysigstartsignatures
\pysiglinewithargsret{\sphinxcode{\sphinxupquote{getter.}}\sphinxbfcode{\sphinxupquote{get\_Leak\_sessions}}}{\sphinxparam{\DUrole{n,n}{wafer\_id}}}{}
\pysigstopsignatures
\sphinxAtStartPar
Used to get all sessions that contain Leak values (for wafer maps)
\begin{quote}\begin{description}
\sphinxlineitem{Parameters}
\sphinxAtStartPar
\sphinxstyleliteralstrong{\sphinxupquote{wafer\_id}} (\sphinxstyleliteralemphasis{\sphinxupquote{\textless{}str\textgreater{}}}) \textendash{} ID of the wafer

\sphinxlineitem{Return \textless{}list\textgreater{}}
\sphinxAtStartPar
List of all sessions that contain Leak values

\end{description}\end{quote}

\end{fulllineitems}

\index{get\_Leak\_structures() (in module getter)@\spxentry{get\_Leak\_structures()}\spxextra{in module getter}}

\begin{fulllineitems}
\phantomsection\label{\detokenize{getter:getter.get_Leak_structures}}
\pysigstartsignatures
\pysiglinewithargsret{\sphinxcode{\sphinxupquote{getter.}}\sphinxbfcode{\sphinxupquote{get\_Leak\_structures}}}{\sphinxparam{\DUrole{n,n}{wafer\_id}}, \sphinxparam{\DUrole{n,n}{session}}}{}
\pysigstopsignatures
\sphinxAtStartPar
Used to get all structures that contain Leak values (for wafer maps)
\begin{quote}\begin{description}
\sphinxlineitem{Parameters}\begin{itemize}
\item {} 
\sphinxAtStartPar
\sphinxstyleliteralstrong{\sphinxupquote{wafer\_id}} (\sphinxstyleliteralemphasis{\sphinxupquote{\textless{}str\textgreater{}}}) \textendash{} ID of the wafer

\item {} 
\sphinxAtStartPar
\sphinxstyleliteralstrong{\sphinxupquote{session}} (\sphinxstyleliteralemphasis{\sphinxupquote{\textless{}str\textgreater{}}}) \textendash{} Name of the session

\end{itemize}

\sphinxlineitem{Return \textless{}list\textgreater{}}
\sphinxAtStartPar
List of all structures that contain Leak values

\end{description}\end{quote}

\end{fulllineitems}

\index{get\_R\_sessions() (in module getter)@\spxentry{get\_R\_sessions()}\spxextra{in module getter}}

\begin{fulllineitems}
\phantomsection\label{\detokenize{getter:getter.get_R_sessions}}
\pysigstartsignatures
\pysiglinewithargsret{\sphinxcode{\sphinxupquote{getter.}}\sphinxbfcode{\sphinxupquote{get\_R\_sessions}}}{\sphinxparam{\DUrole{n,n}{wafer\_id}}}{}
\pysigstopsignatures
\sphinxAtStartPar
Used to get all sessions that contain R values (for wafer maps)
\begin{quote}\begin{description}
\sphinxlineitem{Parameters}
\sphinxAtStartPar
\sphinxstyleliteralstrong{\sphinxupquote{wafer\_id}} (\sphinxstyleliteralemphasis{\sphinxupquote{\textless{}str\textgreater{}}}) \textendash{} ID of the wafer

\sphinxlineitem{Return \textless{}list\textgreater{}}
\sphinxAtStartPar
List of all sessions that contain R values

\end{description}\end{quote}

\end{fulllineitems}

\index{get\_R\_structures() (in module getter)@\spxentry{get\_R\_structures()}\spxextra{in module getter}}

\begin{fulllineitems}
\phantomsection\label{\detokenize{getter:getter.get_R_structures}}
\pysigstartsignatures
\pysiglinewithargsret{\sphinxcode{\sphinxupquote{getter.}}\sphinxbfcode{\sphinxupquote{get\_R\_structures}}}{\sphinxparam{\DUrole{n,n}{wafer\_id}}, \sphinxparam{\DUrole{n,n}{session}}}{}
\pysigstopsignatures
\sphinxAtStartPar
Used to get all structures that contain R values (for wafer maps)
\begin{quote}\begin{description}
\sphinxlineitem{Parameters}\begin{itemize}
\item {} 
\sphinxAtStartPar
\sphinxstyleliteralstrong{\sphinxupquote{wafer\_id}} (\sphinxstyleliteralemphasis{\sphinxupquote{\textless{}str\textgreater{}}}) \textendash{} ID of the wafer

\item {} 
\sphinxAtStartPar
\sphinxstyleliteralstrong{\sphinxupquote{session}} (\sphinxstyleliteralemphasis{\sphinxupquote{\textless{}str\textgreater{}}}) \textendash{} Name of the session

\end{itemize}

\sphinxlineitem{Return \textless{}list\textgreater{}}
\sphinxAtStartPar
List of all structures that contain R values

\end{description}\end{quote}

\end{fulllineitems}

\index{get\_compliance() (in module getter)@\spxentry{get\_compliance()}\spxextra{in module getter}}

\begin{fulllineitems}
\phantomsection\label{\detokenize{getter:getter.get_compliance}}
\pysigstartsignatures
\pysiglinewithargsret{\sphinxcode{\sphinxupquote{getter.}}\sphinxbfcode{\sphinxupquote{get\_compliance}}}{\sphinxparam{\DUrole{n,n}{wafer\_id}}, \sphinxparam{\DUrole{n,n}{session}}}{}
\pysigstopsignatures
\sphinxAtStartPar
This function finds the compliance from the specified structure in the database
Returns None if the structure has no compliance registered
\begin{quote}\begin{description}
\sphinxlineitem{Parameters}\begin{itemize}
\item {} 
\sphinxAtStartPar
\sphinxstyleliteralstrong{\sphinxupquote{wafer\_id}} (\sphinxstyleliteralemphasis{\sphinxupquote{\textless{}str\textgreater{}}}) \textendash{} name of the wafer\_id

\item {} 
\sphinxAtStartPar
\sphinxstyleliteralstrong{\sphinxupquote{session}} (\sphinxstyleliteralemphasis{\sphinxupquote{\textless{}str\textgreater{}}}) \textendash{} name of the session

\end{itemize}

\sphinxlineitem{Return \textless{}str\textgreater{}}
\sphinxAtStartPar
the compliance in the wafer

\end{description}\end{quote}

\end{fulllineitems}

\index{get\_coords() (in module getter)@\spxentry{get\_coords()}\spxextra{in module getter}}

\begin{fulllineitems}
\phantomsection\label{\detokenize{getter:getter.get_coords}}
\pysigstartsignatures
\pysiglinewithargsret{\sphinxcode{\sphinxupquote{getter.}}\sphinxbfcode{\sphinxupquote{get\_coords}}}{\sphinxparam{\DUrole{n,n}{wafer\_id}}}{}
\pysigstopsignatures
\sphinxAtStartPar
This function finds all the coordinates from the specified wafer in the database
\begin{quote}\begin{description}
\sphinxlineitem{Parameters}
\sphinxAtStartPar
\sphinxstyleliteralstrong{\sphinxupquote{wafer\_id}} (\sphinxstyleliteralemphasis{\sphinxupquote{\textless{}str\textgreater{}}}) \textendash{} name of the wafer\_id

\sphinxlineitem{Return \textless{}list\textgreater{}}
\sphinxAtStartPar
the list of coordinates in the wafer

\end{description}\end{quote}

\end{fulllineitems}

\index{get\_db\_name() (in module getter)@\spxentry{get\_db\_name()}\spxextra{in module getter}}

\begin{fulllineitems}
\phantomsection\label{\detokenize{getter:getter.get_db_name}}
\pysigstartsignatures
\pysiglinewithargsret{\sphinxcode{\sphinxupquote{getter.}}\sphinxbfcode{\sphinxupquote{get\_db\_name}}}{\sphinxparam{\DUrole{n,n}{db\_name}\DUrole{o,o}{=}\DUrole{default_value}{\textquotesingle{}New Wafers\textquotesingle{}}}}{}
\pysigstopsignatures
\sphinxAtStartPar
Used to know which database has to be opened. Default parameter can be changed manually, so a new database will be created if it doesn’t exist yet.
\begin{quote}\begin{description}
\sphinxlineitem{Parameters}
\sphinxAtStartPar
\sphinxstyleliteralstrong{\sphinxupquote{db\_name}} (\sphinxstyleliteralemphasis{\sphinxupquote{\textless{}str\textgreater{}}}) \textendash{} Name of the database. Please change it to create a new database or switch to another existing

\sphinxlineitem{Returns}
\sphinxAtStartPar
Name of the database

\end{description}\end{quote}

\end{fulllineitems}

\index{get\_filenames() (in module getter)@\spxentry{get\_filenames()}\spxextra{in module getter}}

\begin{fulllineitems}
\phantomsection\label{\detokenize{getter:getter.get_filenames}}
\pysigstartsignatures
\pysiglinewithargsret{\sphinxcode{\sphinxupquote{getter.}}\sphinxbfcode{\sphinxupquote{get\_filenames}}}{\sphinxparam{\DUrole{n,n}{wafer\_id}}}{}
\pysigstopsignatures
\sphinxAtStartPar
This function finds all the filenames in the specified wafer in the database
\begin{quote}\begin{description}
\sphinxlineitem{Parameters}
\sphinxAtStartPar
\sphinxstyleliteralstrong{\sphinxupquote{wafer\_id}} (\sphinxstyleliteralemphasis{\sphinxupquote{\textless{}str\textgreater{}}}) \textendash{} name of the wafer\_id

\sphinxlineitem{Return \textless{}list\textgreater{}}
\sphinxAtStartPar
the list of filenames in the wafer

\end{description}\end{quote}

\end{fulllineitems}

\index{get\_map\_sessions() (in module getter)@\spxentry{get\_map\_sessions()}\spxextra{in module getter}}

\begin{fulllineitems}
\phantomsection\label{\detokenize{getter:getter.get_map_sessions}}
\pysigstartsignatures
\pysiglinewithargsret{\sphinxcode{\sphinxupquote{getter.}}\sphinxbfcode{\sphinxupquote{get\_map\_sessions}}}{\sphinxparam{\DUrole{n,n}{wafer\_id}}}{}
\pysigstopsignatures
\sphinxAtStartPar
Used to get all sessions that contain I\sphinxhyphen{}V measurements (for wafer maps) inside a wafer.
\begin{quote}\begin{description}
\sphinxlineitem{Parameters}
\sphinxAtStartPar
\sphinxstyleliteralstrong{\sphinxupquote{wafer\_id}} (\sphinxstyleliteralemphasis{\sphinxupquote{\textless{}str\textgreater{}}}) \textendash{} ID of the wafer

\sphinxlineitem{Return \textless{}list of str\textgreater{}}
\sphinxAtStartPar
All sessions with I\sphinxhyphen{}V measurements inside the wafer

\end{description}\end{quote}

\end{fulllineitems}

\index{get\_map\_structures() (in module getter)@\spxentry{get\_map\_structures()}\spxextra{in module getter}}

\begin{fulllineitems}
\phantomsection\label{\detokenize{getter:getter.get_map_structures}}
\pysigstartsignatures
\pysiglinewithargsret{\sphinxcode{\sphinxupquote{getter.}}\sphinxbfcode{\sphinxupquote{get\_map\_structures}}}{\sphinxparam{\DUrole{n,n}{wafer\_id}}, \sphinxparam{\DUrole{n,n}{session}}}{}
\pysigstopsignatures
\sphinxAtStartPar
Used to get all structures that contain I\sphinxhyphen{}V measurements (for wafer maps) inside the given session of a wafer.
\begin{quote}\begin{description}
\sphinxlineitem{Parameters}\begin{itemize}
\item {} 
\sphinxAtStartPar
\sphinxstyleliteralstrong{\sphinxupquote{wafer\_id}} (\sphinxstyleliteralemphasis{\sphinxupquote{\textless{}str\textgreater{}}}) \textendash{} ID of the wafer

\item {} 
\sphinxAtStartPar
\sphinxstyleliteralstrong{\sphinxupquote{session}} (\sphinxstyleliteralemphasis{\sphinxupquote{\textless{}str\textgreater{}}}) \textendash{} Selected session

\end{itemize}

\sphinxlineitem{Return \textless{}list of str\textgreater{}}
\sphinxAtStartPar
All structures that contain I\sphinxhyphen{}V measurements inside the session

\end{description}\end{quote}

\end{fulllineitems}

\index{get\_matrices\_with\_I() (in module getter)@\spxentry{get\_matrices\_with\_I()}\spxextra{in module getter}}

\begin{fulllineitems}
\phantomsection\label{\detokenize{getter:getter.get_matrices_with_I}}
\pysigstartsignatures
\pysiglinewithargsret{\sphinxcode{\sphinxupquote{getter.}}\sphinxbfcode{\sphinxupquote{get\_matrices\_with\_I}}}{\sphinxparam{\DUrole{n,n}{wafer\_id}}, \sphinxparam{\DUrole{n,n}{structure\_id}}}{}
\pysigstopsignatures
\sphinxAtStartPar
This function finds all matrices that contain I\sphinxhyphen{}V measurements. Used to display buttons in the right place in the UI
\begin{quote}\begin{description}
\sphinxlineitem{Parameters}\begin{itemize}
\item {} 
\sphinxAtStartPar
\sphinxstyleliteralstrong{\sphinxupquote{wafer\_id}} (\sphinxstyleliteralemphasis{\sphinxupquote{\textless{}str\textgreater{}}}) \textendash{} the name of the wafer

\item {} 
\sphinxAtStartPar
\sphinxstyleliteralstrong{\sphinxupquote{structure\_id}} (\sphinxstyleliteralemphasis{\sphinxupquote{\textless{}str\textgreater{}}}) \textendash{} the name of the structure

\end{itemize}

\sphinxlineitem{Return \textless{}list\textgreater{}}
\sphinxAtStartPar
List of matrices that contains I\sphinxhyphen{}V measurements

\end{description}\end{quote}

\end{fulllineitems}

\index{get\_sessions() (in module getter)@\spxentry{get\_sessions()}\spxextra{in module getter}}

\begin{fulllineitems}
\phantomsection\label{\detokenize{getter:getter.get_sessions}}
\pysigstartsignatures
\pysiglinewithargsret{\sphinxcode{\sphinxupquote{getter.}}\sphinxbfcode{\sphinxupquote{get\_sessions}}}{\sphinxparam{\DUrole{n,n}{wafer\_id}}}{}
\pysigstopsignatures
\sphinxAtStartPar
Used to get all sessions inside a given wafer.
\begin{quote}\begin{description}
\sphinxlineitem{Parameters}
\sphinxAtStartPar
\sphinxstyleliteralstrong{\sphinxupquote{wafer\_id}} (\sphinxstyleliteralemphasis{\sphinxupquote{\textless{}str\textgreater{}}}) \textendash{} ID of the wafer

\sphinxlineitem{Return \textless{}list of str\textgreater{}}
\sphinxAtStartPar
All sessions inside the wafer

\end{description}\end{quote}

\end{fulllineitems}

\index{get\_structures() (in module getter)@\spxentry{get\_structures()}\spxextra{in module getter}}

\begin{fulllineitems}
\phantomsection\label{\detokenize{getter:getter.get_structures}}
\pysigstartsignatures
\pysiglinewithargsret{\sphinxcode{\sphinxupquote{getter.}}\sphinxbfcode{\sphinxupquote{get\_structures}}}{\sphinxparam{\DUrole{n,n}{wafer\_id}}, \sphinxparam{\DUrole{n,n}{session}}}{}
\pysigstopsignatures
\sphinxAtStartPar
Used to get all structures  inside the given session of a wafer.
\begin{quote}\begin{description}
\sphinxlineitem{Parameters}\begin{itemize}
\item {} 
\sphinxAtStartPar
\sphinxstyleliteralstrong{\sphinxupquote{wafer\_id}} (\sphinxstyleliteralemphasis{\sphinxupquote{\textless{}str\textgreater{}}}) \textendash{} ID of the wafer

\item {} 
\sphinxAtStartPar
\sphinxstyleliteralstrong{\sphinxupquote{session}} (\sphinxstyleliteralemphasis{\sphinxupquote{\textless{}str\textgreater{}}}) \textendash{} Selected session

\end{itemize}

\sphinxlineitem{Return \textless{}list of str\textgreater{}}
\sphinxAtStartPar
All structures inside the session

\end{description}\end{quote}

\end{fulllineitems}

\index{get\_temps() (in module getter)@\spxentry{get\_temps()}\spxextra{in module getter}}

\begin{fulllineitems}
\phantomsection\label{\detokenize{getter:getter.get_temps}}
\pysigstartsignatures
\pysiglinewithargsret{\sphinxcode{\sphinxupquote{getter.}}\sphinxbfcode{\sphinxupquote{get\_temps}}}{\sphinxparam{\DUrole{n,n}{wafer\_id}}}{}
\pysigstopsignatures
\sphinxAtStartPar
This function finds all the temperatures from the specified wafer in the database
\begin{quote}\begin{description}
\sphinxlineitem{Parameters}
\sphinxAtStartPar
\sphinxstyleliteralstrong{\sphinxupquote{wafer\_id}} (\sphinxstyleliteralemphasis{\sphinxupquote{\textless{}str\textgreater{}}}) \textendash{} name of the wafer\_id

\sphinxlineitem{Return \textless{}list\textgreater{}}
\sphinxAtStartPar
the list of temperatures in the wafer

\end{description}\end{quote}

\end{fulllineitems}

\index{get\_types() (in module getter)@\spxentry{get\_types()}\spxextra{in module getter}}

\begin{fulllineitems}
\phantomsection\label{\detokenize{getter:getter.get_types}}
\pysigstartsignatures
\pysiglinewithargsret{\sphinxcode{\sphinxupquote{getter.}}\sphinxbfcode{\sphinxupquote{get\_types}}}{\sphinxparam{\DUrole{n,n}{wafer\_id}}}{}
\pysigstopsignatures
\sphinxAtStartPar
This function finds all the types of measurements from the specified wafer in the database
\begin{quote}\begin{description}
\sphinxlineitem{Parameters}
\sphinxAtStartPar
\sphinxstyleliteralstrong{\sphinxupquote{wafer\_id}} (\sphinxstyleliteralemphasis{\sphinxupquote{\textless{}str\textgreater{}}}) \textendash{} name of the wafer\_id

\sphinxlineitem{Return \textless{}list\textgreater{}}
\sphinxAtStartPar
the list of types in the wafer

\end{description}\end{quote}

\end{fulllineitems}

\index{get\_wafer() (in module getter)@\spxentry{get\_wafer()}\spxextra{in module getter}}

\begin{fulllineitems}
\phantomsection\label{\detokenize{getter:getter.get_wafer}}
\pysigstartsignatures
\pysiglinewithargsret{\sphinxcode{\sphinxupquote{getter.}}\sphinxbfcode{\sphinxupquote{get\_wafer}}}{\sphinxparam{\DUrole{n,n}{wafer\_id}}}{}
\pysigstopsignatures
\sphinxAtStartPar
This function finds the wafer specified in the database. /!Only use to read data, not to write data in the DB /!
:param \textless{}str\textgreater{} wafer\_id: name of the wafer\_id
\begin{quote}\begin{description}
\sphinxlineitem{Return \textless{}dict\textgreater{}}
\sphinxAtStartPar
the wafer

\end{description}\end{quote}

\end{fulllineitems}


\sphinxstepscope


\chapter{wafer\sphinxhyphen{}management}
\label{\detokenize{modules:wafer-management}}\label{\detokenize{modules::doc}}
\sphinxstepscope


\section{new\_manage\_DB module}
\label{\detokenize{new_manage_DB:module-new_manage_DB}}\label{\detokenize{new_manage_DB:new-manage-db-module}}\label{\detokenize{new_manage_DB::doc}}\index{module@\spxentry{module}!new\_manage\_DB@\spxentry{new\_manage\_DB}}\index{new\_manage\_DB@\spxentry{new\_manage\_DB}!module@\spxentry{module}}\index{create\_db() (in module new\_manage\_DB)@\spxentry{create\_db()}\spxextra{in module new\_manage\_DB}}

\begin{fulllineitems}
\phantomsection\label{\detokenize{new_manage_DB:new_manage_DB.create_db}}
\pysigstartsignatures
\pysiglinewithargsret{\sphinxcode{\sphinxupquote{new\_manage\_DB.}}\sphinxbfcode{\sphinxupquote{create\_db}}}{\sphinxparam{\DUrole{n,n}{path}}, \sphinxparam{\DUrole{n,n}{is\_JV}}}{}
\pysigstopsignatures
\sphinxAtStartPar
Used to get information from .txt files.
This function creates a database or open it if it already exists, and fill it with measurement information
We put all the file’s information in a dictionary, and then we write all the dictionary in the database, so we just call the db once.
This is much faster.
Also, indexes are created if they don’t already exist, so searching in the database in faster
\begin{quote}\begin{description}
\sphinxlineitem{Parameters}\begin{itemize}
\item {} 
\sphinxAtStartPar
\sphinxstyleliteralstrong{\sphinxupquote{path}} (\sphinxstyleliteralemphasis{\sphinxupquote{\textless{}str\textgreater{}}}) \textendash{} path of the file

\item {} 
\sphinxAtStartPar
\sphinxstyleliteralstrong{\sphinxupquote{is\_JV}} (\sphinxstyleliteralemphasis{\sphinxupquote{\textless{}bool\textgreater{}}}) \textendash{} True if the user wants to register J\sphinxhyphen{}V measurements, False otherwise

\end{itemize}

\sphinxlineitem{Return \textless{}list\textgreater{}}
\sphinxAtStartPar
a list containing all wafers that have been registered

\end{description}\end{quote}

\end{fulllineitems}

\index{create\_db\_it() (in module new\_manage\_DB)@\spxentry{create\_db\_it()}\spxextra{in module new\_manage\_DB}}

\begin{fulllineitems}
\phantomsection\label{\detokenize{new_manage_DB:new_manage_DB.create_db_it}}
\pysigstartsignatures
\pysiglinewithargsret{\sphinxcode{\sphinxupquote{new\_manage\_DB.}}\sphinxbfcode{\sphinxupquote{create\_db\_it}}}{\sphinxparam{\DUrole{n,n}{path}}}{}
\pysigstopsignatures
\sphinxAtStartPar
Used to get information from .txt files that contain data for It measurements.
This function creates a database or open it if it already exists, and fill it with measurement information
We put all the file’s information in a dictionary, and then we write all the dictionary in the database, so we just call the db once.
This is much faster.
Also, indexes are created if they don’t already exist, so searching in the database in faster
\begin{quote}\begin{description}
\sphinxlineitem{Parameters}
\sphinxAtStartPar
\sphinxstyleliteralstrong{\sphinxupquote{path}} (\sphinxstyleliteralemphasis{\sphinxupquote{\textless{}str\textgreater{}}}) \textendash{} path of the file

\sphinxlineitem{Return \textless{}list\textgreater{}}
\sphinxAtStartPar
a list containing all wafers that have been registered

\end{description}\end{quote}

\end{fulllineitems}

\index{create\_db\_lim() (in module new\_manage\_DB)@\spxentry{create\_db\_lim()}\spxextra{in module new\_manage\_DB}}

\begin{fulllineitems}
\phantomsection\label{\detokenize{new_manage_DB:new_manage_DB.create_db_lim}}
\pysigstartsignatures
\pysiglinewithargsret{\sphinxcode{\sphinxupquote{new\_manage\_DB.}}\sphinxbfcode{\sphinxupquote{create\_db\_lim}}}{\sphinxparam{\DUrole{n,n}{path}}}{}
\pysigstopsignatures
\sphinxAtStartPar
Used to get information from .lim files.
First, we read the lim file and get all information from it in a list.
Then, we read the file without extension and merge information from both files.
Finally, we write it in the database.
\begin{quote}\begin{description}
\sphinxlineitem{Parameters}
\sphinxAtStartPar
\sphinxstyleliteralstrong{\sphinxupquote{path}} (\sphinxstyleliteralemphasis{\sphinxupquote{\textless{}str\textgreater{}}}) \textendash{} path of the file

\end{description}\end{quote}

\end{fulllineitems}

\index{create\_db\_tbl() (in module new\_manage\_DB)@\spxentry{create\_db\_tbl()}\spxextra{in module new\_manage\_DB}}

\begin{fulllineitems}
\phantomsection\label{\detokenize{new_manage_DB:new_manage_DB.create_db_tbl}}
\pysigstartsignatures
\pysiglinewithargsret{\sphinxcode{\sphinxupquote{new\_manage\_DB.}}\sphinxbfcode{\sphinxupquote{create\_db\_tbl}}}{\sphinxparam{\DUrole{n,n}{path}}, \sphinxparam{\DUrole{n,n}{is\_JV}}}{}
\pysigstopsignatures
\sphinxAtStartPar
Used to get information from .tbl files.
This function creates a database or open it if it already exists, and fill it with measurement information
We put all the file’s information in a dictionary, and then we write all the dictionary in the database, so we just call the db once.
This is much faster.
Also, indexes are created if they don’t already exist, so searching in the database in faster
\begin{quote}\begin{description}
\sphinxlineitem{Parameters}\begin{itemize}
\item {} 
\sphinxAtStartPar
\sphinxstyleliteralstrong{\sphinxupquote{path}} (\sphinxstyleliteralemphasis{\sphinxupquote{\textless{}str\textgreater{}}}) \textendash{} path of the file

\item {} 
\sphinxAtStartPar
\sphinxstyleliteralstrong{\sphinxupquote{is\_JV}} (\sphinxstyleliteralemphasis{\sphinxupquote{\textless{}bool\textgreater{}}}) \textendash{} True if the user wants to register J\sphinxhyphen{}V measurements, False otherwise

\end{itemize}

\sphinxlineitem{Return \textless{}list\textgreater{}}
\sphinxAtStartPar
a list containing all wafers that have been registered

\end{description}\end{quote}

\end{fulllineitems}

\index{setCompliance() (in module new\_manage\_DB)@\spxentry{setCompliance()}\spxextra{in module new\_manage\_DB}}

\begin{fulllineitems}
\phantomsection\label{\detokenize{new_manage_DB:new_manage_DB.setCompliance}}
\pysigstartsignatures
\pysiglinewithargsret{\sphinxcode{\sphinxupquote{new\_manage\_DB.}}\sphinxbfcode{\sphinxupquote{setCompliance}}}{\sphinxparam{\DUrole{n,n}{waferId}}, \sphinxparam{\DUrole{n,n}{session}}, \sphinxparam{\DUrole{n,n}{compliance}}}{}
\pysigstopsignatures
\end{fulllineitems}


\sphinxstepscope


\section{normal\_plots module}
\label{\detokenize{normal_plots:module-normal_plots}}\label{\detokenize{normal_plots:normal-plots-module}}\label{\detokenize{normal_plots::doc}}\index{module@\spxentry{module}!normal\_plots@\spxentry{normal\_plots}}\index{normal\_plots@\spxentry{normal\_plots}!module@\spxentry{module}}\index{C\_normal\_distrib\_neg() (in module normal\_plots)@\spxentry{C\_normal\_distrib\_neg()}\spxextra{in module normal\_plots}}

\begin{fulllineitems}
\phantomsection\label{\detokenize{normal_plots:normal_plots.C_normal_distrib_neg}}
\pysigstartsignatures
\pysiglinewithargsret{\sphinxcode{\sphinxupquote{normal\_plots.}}\sphinxbfcode{\sphinxupquote{C\_normal\_distrib\_neg}}}{\sphinxparam{\DUrole{n,n}{wafer\_id}}, \sphinxparam{\DUrole{n,n}{sessions}}, \sphinxparam{\DUrole{n,n}{structures}}, \sphinxparam{\DUrole{n,n}{dies}}}{}
\pysigstopsignatures
\sphinxAtStartPar
Used to plot the normal distribution of negatives values of C inside a wafer, following selected filters.
We first get data inside the database and then plot it in a figure. We use a probability scale, and plot the reference line and the 90\% confidence interval
\begin{quote}\begin{description}
\sphinxlineitem{Parameters}\begin{itemize}
\item {} 
\sphinxAtStartPar
\sphinxstyleliteralstrong{\sphinxupquote{wafer\_id}} \textendash{} ID of the wafer

\item {} 
\sphinxAtStartPar
\sphinxstyleliteralstrong{\sphinxupquote{sessions}} \textendash{} Selected sessions

\item {} 
\sphinxAtStartPar
\sphinxstyleliteralstrong{\sphinxupquote{structures}} \textendash{} Selected structures

\item {} 
\sphinxAtStartPar
\sphinxstyleliteralstrong{\sphinxupquote{dies}} \textendash{} Selected dies

\end{itemize}

\sphinxlineitem{Returns}
\sphinxAtStartPar
The plot, converted into base64

\end{description}\end{quote}

\end{fulllineitems}

\index{C\_normal\_distrib\_pos() (in module normal\_plots)@\spxentry{C\_normal\_distrib\_pos()}\spxextra{in module normal\_plots}}

\begin{fulllineitems}
\phantomsection\label{\detokenize{normal_plots:normal_plots.C_normal_distrib_pos}}
\pysigstartsignatures
\pysiglinewithargsret{\sphinxcode{\sphinxupquote{normal\_plots.}}\sphinxbfcode{\sphinxupquote{C\_normal\_distrib\_pos}}}{\sphinxparam{\DUrole{n,n}{wafer\_id}}, \sphinxparam{\DUrole{n,n}{sessions}}, \sphinxparam{\DUrole{n,n}{structures}}, \sphinxparam{\DUrole{n,n}{dies}}}{}
\pysigstopsignatures
\sphinxAtStartPar
Used to plot the normal distribution of positive values of C inside a wafer, following selected filters.
We first get data inside the database and then plot it in a figure. We use a probability scale, and plot the reference line and the 90\% confidence interval
\begin{quote}\begin{description}
\sphinxlineitem{Parameters}\begin{itemize}
\item {} 
\sphinxAtStartPar
\sphinxstyleliteralstrong{\sphinxupquote{wafer\_id}} \textendash{} ID of the wafer

\item {} 
\sphinxAtStartPar
\sphinxstyleliteralstrong{\sphinxupquote{sessions}} \textendash{} Selected sessions

\item {} 
\sphinxAtStartPar
\sphinxstyleliteralstrong{\sphinxupquote{structures}} \textendash{} Selected structures

\item {} 
\sphinxAtStartPar
\sphinxstyleliteralstrong{\sphinxupquote{dies}} \textendash{} Selected dies

\end{itemize}

\sphinxlineitem{Returns}
\sphinxAtStartPar
The plot, converted into base64

\end{description}\end{quote}

\end{fulllineitems}

\index{Cmes\_normal\_distrib\_neg() (in module normal\_plots)@\spxentry{Cmes\_normal\_distrib\_neg()}\spxextra{in module normal\_plots}}

\begin{fulllineitems}
\phantomsection\label{\detokenize{normal_plots:normal_plots.Cmes_normal_distrib_neg}}
\pysigstartsignatures
\pysiglinewithargsret{\sphinxcode{\sphinxupquote{normal\_plots.}}\sphinxbfcode{\sphinxupquote{Cmes\_normal\_distrib\_neg}}}{\sphinxparam{\DUrole{n,n}{wafer\_id}}, \sphinxparam{\DUrole{n,n}{sessions}}, \sphinxparam{\DUrole{n,n}{structures}}, \sphinxparam{\DUrole{n,n}{dies}}}{}
\pysigstopsignatures
\sphinxAtStartPar
Used to plot the normal distribution of negative values of Cmes inside a wafer, following selected filters.
We first get data inside the database and then plot it in a figure. We use a probability scale, and plot the reference line and the 90\% confidence interval
\begin{quote}\begin{description}
\sphinxlineitem{Parameters}\begin{itemize}
\item {} 
\sphinxAtStartPar
\sphinxstyleliteralstrong{\sphinxupquote{wafer\_id}} \textendash{} ID of the wafer

\item {} 
\sphinxAtStartPar
\sphinxstyleliteralstrong{\sphinxupquote{sessions}} \textendash{} Selected sessions

\item {} 
\sphinxAtStartPar
\sphinxstyleliteralstrong{\sphinxupquote{structures}} \textendash{} Selected structures

\item {} 
\sphinxAtStartPar
\sphinxstyleliteralstrong{\sphinxupquote{dies}} \textendash{} Selected dies

\end{itemize}

\sphinxlineitem{Returns}
\sphinxAtStartPar
The plot, converted into base64

\end{description}\end{quote}

\end{fulllineitems}

\index{Cmes\_normal\_distrib\_pos() (in module normal\_plots)@\spxentry{Cmes\_normal\_distrib\_pos()}\spxextra{in module normal\_plots}}

\begin{fulllineitems}
\phantomsection\label{\detokenize{normal_plots:normal_plots.Cmes_normal_distrib_pos}}
\pysigstartsignatures
\pysiglinewithargsret{\sphinxcode{\sphinxupquote{normal\_plots.}}\sphinxbfcode{\sphinxupquote{Cmes\_normal\_distrib\_pos}}}{\sphinxparam{\DUrole{n,n}{wafer\_id}}, \sphinxparam{\DUrole{n,n}{sessions}}, \sphinxparam{\DUrole{n,n}{structures}}, \sphinxparam{\DUrole{n,n}{dies}}}{}
\pysigstopsignatures
\sphinxAtStartPar
Used to plot the normal distribution of positive values of Cmes inside a wafer, following selected filters.
We first get data inside the database and then plot it in a figure. We use a probability scale, and plot the reference line and the 90\% confidence interval
\begin{quote}\begin{description}
\sphinxlineitem{Parameters}\begin{itemize}
\item {} 
\sphinxAtStartPar
\sphinxstyleliteralstrong{\sphinxupquote{wafer\_id}} \textendash{} ID of the wafer

\item {} 
\sphinxAtStartPar
\sphinxstyleliteralstrong{\sphinxupquote{sessions}} \textendash{} Selected sessions

\item {} 
\sphinxAtStartPar
\sphinxstyleliteralstrong{\sphinxupquote{structures}} \textendash{} Selected structures

\item {} 
\sphinxAtStartPar
\sphinxstyleliteralstrong{\sphinxupquote{dies}} \textendash{} Selected dies

\end{itemize}

\sphinxlineitem{Returns}
\sphinxAtStartPar
The plot, converted into base64

\end{description}\end{quote}

\end{fulllineitems}

\index{Leakage\_normal\_distrib\_neg() (in module normal\_plots)@\spxentry{Leakage\_normal\_distrib\_neg()}\spxextra{in module normal\_plots}}

\begin{fulllineitems}
\phantomsection\label{\detokenize{normal_plots:normal_plots.Leakage_normal_distrib_neg}}
\pysigstartsignatures
\pysiglinewithargsret{\sphinxcode{\sphinxupquote{normal\_plots.}}\sphinxbfcode{\sphinxupquote{Leakage\_normal\_distrib\_neg}}}{\sphinxparam{\DUrole{n,n}{wafer\_id}}, \sphinxparam{\DUrole{n,n}{sessions}}, \sphinxparam{\DUrole{n,n}{structures}}, \sphinxparam{\DUrole{n,n}{dies}}}{}
\pysigstopsignatures
\sphinxAtStartPar
Used to plot the normal distribution of negative values of Leakage inside a wafer, following selected filters.
We first get data inside the database and then plot it in a figure. We use a probability scale, and plot the reference line and the 90\% confidence interval
\begin{quote}\begin{description}
\sphinxlineitem{Parameters}\begin{itemize}
\item {} 
\sphinxAtStartPar
\sphinxstyleliteralstrong{\sphinxupquote{wafer\_id}} \textendash{} ID of the wafer

\item {} 
\sphinxAtStartPar
\sphinxstyleliteralstrong{\sphinxupquote{sessions}} \textendash{} Selected sessions

\item {} 
\sphinxAtStartPar
\sphinxstyleliteralstrong{\sphinxupquote{structures}} \textendash{} Selected structures

\item {} 
\sphinxAtStartPar
\sphinxstyleliteralstrong{\sphinxupquote{dies}} \textendash{} Selected dies

\end{itemize}

\sphinxlineitem{Returns}
\sphinxAtStartPar
The plot, converted into base64

\end{description}\end{quote}

\end{fulllineitems}

\index{Leakage\_normal\_distrib\_pos() (in module normal\_plots)@\spxentry{Leakage\_normal\_distrib\_pos()}\spxextra{in module normal\_plots}}

\begin{fulllineitems}
\phantomsection\label{\detokenize{normal_plots:normal_plots.Leakage_normal_distrib_pos}}
\pysigstartsignatures
\pysiglinewithargsret{\sphinxcode{\sphinxupquote{normal\_plots.}}\sphinxbfcode{\sphinxupquote{Leakage\_normal\_distrib\_pos}}}{\sphinxparam{\DUrole{n,n}{wafer\_id}}, \sphinxparam{\DUrole{n,n}{sessions}}, \sphinxparam{\DUrole{n,n}{structures}}, \sphinxparam{\DUrole{n,n}{dies}}}{}
\pysigstopsignatures
\sphinxAtStartPar
Used to plot the normal distribution of positive values of Leakage inside a wafer, following selected filters.
We first get data inside the database and then plot it in a figure. We use a probability scale, and plot the reference line and the 90\% confidence interval
\begin{quote}\begin{description}
\sphinxlineitem{Parameters}\begin{itemize}
\item {} 
\sphinxAtStartPar
\sphinxstyleliteralstrong{\sphinxupquote{wafer\_id}} \textendash{} ID of the wafer

\item {} 
\sphinxAtStartPar
\sphinxstyleliteralstrong{\sphinxupquote{sessions}} \textendash{} Selected sessions

\item {} 
\sphinxAtStartPar
\sphinxstyleliteralstrong{\sphinxupquote{structures}} \textendash{} Selected structures

\item {} 
\sphinxAtStartPar
\sphinxstyleliteralstrong{\sphinxupquote{dies}} \textendash{} Selected dies

\end{itemize}

\sphinxlineitem{Returns}
\sphinxAtStartPar
The plot, converted into base64

\end{description}\end{quote}

\end{fulllineitems}

\index{R\_normal\_distrib\_neg() (in module normal\_plots)@\spxentry{R\_normal\_distrib\_neg()}\spxextra{in module normal\_plots}}

\begin{fulllineitems}
\phantomsection\label{\detokenize{normal_plots:normal_plots.R_normal_distrib_neg}}
\pysigstartsignatures
\pysiglinewithargsret{\sphinxcode{\sphinxupquote{normal\_plots.}}\sphinxbfcode{\sphinxupquote{R\_normal\_distrib\_neg}}}{\sphinxparam{\DUrole{n,n}{wafer\_id}}, \sphinxparam{\DUrole{n,n}{sessions}}, \sphinxparam{\DUrole{n,n}{structures}}, \sphinxparam{\DUrole{n,n}{dies}}}{}
\pysigstopsignatures
\sphinxAtStartPar
Used to plot the normal distribution of negative values of R inside a wafer, following selected filters.
We first get data inside the database and then plot it in a figure. We use a probability scale, and plot the reference line and the 90\% confidence interval
\begin{quote}\begin{description}
\sphinxlineitem{Parameters}\begin{itemize}
\item {} 
\sphinxAtStartPar
\sphinxstyleliteralstrong{\sphinxupquote{wafer\_id}} \textendash{} ID of the wafer

\item {} 
\sphinxAtStartPar
\sphinxstyleliteralstrong{\sphinxupquote{sessions}} \textendash{} Selected sessions

\item {} 
\sphinxAtStartPar
\sphinxstyleliteralstrong{\sphinxupquote{structures}} \textendash{} Selected structures

\item {} 
\sphinxAtStartPar
\sphinxstyleliteralstrong{\sphinxupquote{dies}} \textendash{} Selected dies

\end{itemize}

\sphinxlineitem{Returns}
\sphinxAtStartPar
The plot, converted into base64

\end{description}\end{quote}

\end{fulllineitems}

\index{R\_normal\_distrib\_pos() (in module normal\_plots)@\spxentry{R\_normal\_distrib\_pos()}\spxextra{in module normal\_plots}}

\begin{fulllineitems}
\phantomsection\label{\detokenize{normal_plots:normal_plots.R_normal_distrib_pos}}
\pysigstartsignatures
\pysiglinewithargsret{\sphinxcode{\sphinxupquote{normal\_plots.}}\sphinxbfcode{\sphinxupquote{R\_normal\_distrib\_pos}}}{\sphinxparam{\DUrole{n,n}{wafer\_id}}, \sphinxparam{\DUrole{n,n}{sessions}}, \sphinxparam{\DUrole{n,n}{structures}}, \sphinxparam{\DUrole{n,n}{dies}}}{}
\pysigstopsignatures
\sphinxAtStartPar
Used to plot the normal distribution of positive values of R inside a wafer, following selected filters.
We first get data inside the database and then plot it in a figure. We use a probability scale, and plot the reference line and the 90\% confidence interval
\begin{quote}\begin{description}
\sphinxlineitem{Parameters}\begin{itemize}
\item {} 
\sphinxAtStartPar
\sphinxstyleliteralstrong{\sphinxupquote{wafer\_id}} \textendash{} ID of the wafer

\item {} 
\sphinxAtStartPar
\sphinxstyleliteralstrong{\sphinxupquote{sessions}} \textendash{} Selected sessions

\item {} 
\sphinxAtStartPar
\sphinxstyleliteralstrong{\sphinxupquote{structures}} \textendash{} Selected structures

\item {} 
\sphinxAtStartPar
\sphinxstyleliteralstrong{\sphinxupquote{dies}} \textendash{} Selected dies

\end{itemize}

\sphinxlineitem{Returns}
\sphinxAtStartPar
The plot, converted into base64

\end{description}\end{quote}

\end{fulllineitems}

\index{VBD\_normal\_distrib\_neg() (in module normal\_plots)@\spxentry{VBD\_normal\_distrib\_neg()}\spxextra{in module normal\_plots}}

\begin{fulllineitems}
\phantomsection\label{\detokenize{normal_plots:normal_plots.VBD_normal_distrib_neg}}
\pysigstartsignatures
\pysiglinewithargsret{\sphinxcode{\sphinxupquote{normal\_plots.}}\sphinxbfcode{\sphinxupquote{VBD\_normal\_distrib\_neg}}}{\sphinxparam{\DUrole{n,n}{wafer\_id}}, \sphinxparam{\DUrole{n,n}{sessions}}, \sphinxparam{\DUrole{n,n}{structures}}, \sphinxparam{\DUrole{n,n}{dies}}}{}
\pysigstopsignatures
\sphinxAtStartPar
Used to plot the normal distribution of negative values of VBD inside a wafer, following selected filters.
We first get data inside the database and then plot it in a figure. We use a probability scale, and plot the reference line and the 90\% confidence interval
\begin{quote}\begin{description}
\sphinxlineitem{Parameters}\begin{itemize}
\item {} 
\sphinxAtStartPar
\sphinxstyleliteralstrong{\sphinxupquote{wafer\_id}} \textendash{} ID of the wafer

\item {} 
\sphinxAtStartPar
\sphinxstyleliteralstrong{\sphinxupquote{sessions}} \textendash{} Selected sessions

\item {} 
\sphinxAtStartPar
\sphinxstyleliteralstrong{\sphinxupquote{structures}} \textendash{} Selected structures

\item {} 
\sphinxAtStartPar
\sphinxstyleliteralstrong{\sphinxupquote{dies}} \textendash{} Selected dies

\end{itemize}

\sphinxlineitem{Returns}
\sphinxAtStartPar
The plot, converted into base64

\end{description}\end{quote}

\end{fulllineitems}

\index{VBD\_normal\_distrib\_pos() (in module normal\_plots)@\spxentry{VBD\_normal\_distrib\_pos()}\spxextra{in module normal\_plots}}

\begin{fulllineitems}
\phantomsection\label{\detokenize{normal_plots:normal_plots.VBD_normal_distrib_pos}}
\pysigstartsignatures
\pysiglinewithargsret{\sphinxcode{\sphinxupquote{normal\_plots.}}\sphinxbfcode{\sphinxupquote{VBD\_normal\_distrib\_pos}}}{\sphinxparam{\DUrole{n,n}{wafer\_id}}, \sphinxparam{\DUrole{n,n}{sessions}}, \sphinxparam{\DUrole{n,n}{structures}}, \sphinxparam{\DUrole{n,n}{dies}}}{}
\pysigstopsignatures
\sphinxAtStartPar
Used to plot the normal distribution of positive values of VBD inside a wafer, following selected filters.
We first get data inside the database and then plot it in a figure. We use a probability scale, and plot the reference line and the 90\% confidence interval
\begin{quote}\begin{description}
\sphinxlineitem{Parameters}\begin{itemize}
\item {} 
\sphinxAtStartPar
\sphinxstyleliteralstrong{\sphinxupquote{wafer\_id}} \textendash{} ID of the wafer

\item {} 
\sphinxAtStartPar
\sphinxstyleliteralstrong{\sphinxupquote{sessions}} \textendash{} Selected sessions

\item {} 
\sphinxAtStartPar
\sphinxstyleliteralstrong{\sphinxupquote{structures}} \textendash{} Selected structures

\item {} 
\sphinxAtStartPar
\sphinxstyleliteralstrong{\sphinxupquote{dies}} \textendash{} Selected dies

\end{itemize}

\sphinxlineitem{Returns}
\sphinxAtStartPar
The plot, converted into base64

\end{description}\end{quote}

\end{fulllineitems}

\index{get\_values() (in module normal\_plots)@\spxentry{get\_values()}\spxextra{in module normal\_plots}}

\begin{fulllineitems}
\phantomsection\label{\detokenize{normal_plots:normal_plots.get_values}}
\pysigstartsignatures
\pysiglinewithargsret{\sphinxcode{\sphinxupquote{normal\_plots.}}\sphinxbfcode{\sphinxupquote{get\_values}}}{\sphinxparam{\DUrole{n,n}{wafer\_id}}}{}
\pysigstopsignatures
\sphinxAtStartPar
Used to know which extracted value is inside a given wafer, so we display only available options to the user
\begin{quote}\begin{description}
\sphinxlineitem{Parameters}
\sphinxAtStartPar
\sphinxstyleliteralstrong{\sphinxupquote{wafer\_id}} \textendash{} ID of the wafer

\sphinxlineitem{Returns}
\sphinxAtStartPar
list of all Extracted values inside the wafer

\end{description}\end{quote}

\end{fulllineitems}


\sphinxstepscope


\section{plot\_and\_powerpoint module}
\label{\detokenize{plot_and_powerpoint:module-plot_and_powerpoint}}\label{\detokenize{plot_and_powerpoint:plot-and-powerpoint-module}}\label{\detokenize{plot_and_powerpoint::doc}}\index{module@\spxentry{module}!plot\_and\_powerpoint@\spxentry{plot\_and\_powerpoint}}\index{plot\_and\_powerpoint@\spxentry{plot\_and\_powerpoint}!module@\spxentry{module}}\index{fig\_to\_base64() (in module plot\_and\_powerpoint)@\spxentry{fig\_to\_base64()}\spxextra{in module plot\_and\_powerpoint}}

\begin{fulllineitems}
\phantomsection\label{\detokenize{plot_and_powerpoint:plot_and_powerpoint.fig_to_base64}}
\pysigstartsignatures
\pysiglinewithargsret{\sphinxcode{\sphinxupquote{plot\_and\_powerpoint.}}\sphinxbfcode{\sphinxupquote{fig\_to\_base64}}}{\sphinxparam{\DUrole{n,n}{fig}}}{}
\pysigstopsignatures
\sphinxAtStartPar
Function used to convert a png to base64 to help communication between server and User. Used in ppt\_matrix.
\begin{quote}\begin{description}
\sphinxlineitem{Parameters}
\sphinxAtStartPar
\sphinxstyleliteralstrong{\sphinxupquote{fig}} (\sphinxstyleliteralemphasis{\sphinxupquote{\textless{}png\textgreater{}}}) \textendash{} a figure in png format

\sphinxlineitem{Return \textless{}base64\textgreater{}}
\sphinxAtStartPar
The converted figure

\end{description}\end{quote}

\end{fulllineitems}

\index{get\_wafer() (in module plot\_and\_powerpoint)@\spxentry{get\_wafer()}\spxextra{in module plot\_and\_powerpoint}}

\begin{fulllineitems}
\phantomsection\label{\detokenize{plot_and_powerpoint:plot_and_powerpoint.get_wafer}}
\pysigstartsignatures
\pysiglinewithargsret{\sphinxcode{\sphinxupquote{plot\_and\_powerpoint.}}\sphinxbfcode{\sphinxupquote{get\_wafer}}}{\sphinxparam{\DUrole{n,n}{wafer\_id}}}{}
\pysigstopsignatures
\sphinxAtStartPar
This function finds the wafer specified in the database
\begin{quote}\begin{description}
\sphinxlineitem{Parameters}
\sphinxAtStartPar
\sphinxstyleliteralstrong{\sphinxupquote{wafer\_id}} (\sphinxstyleliteralemphasis{\sphinxupquote{\textless{}str\textgreater{}}}) \textendash{} name of the wafer\_id

\sphinxlineitem{Return \textless{}dict\textgreater{}}
\sphinxAtStartPar
the wafer

\end{description}\end{quote}

\end{fulllineitems}

\index{plot\_wanted\_matrices() (in module plot\_and\_powerpoint)@\spxentry{plot\_wanted\_matrices()}\spxextra{in module plot\_and\_powerpoint}}

\begin{fulllineitems}
\phantomsection\label{\detokenize{plot_and_powerpoint:plot_and_powerpoint.plot_wanted_matrices}}
\pysigstartsignatures
\pysiglinewithargsret{\sphinxcode{\sphinxupquote{plot\_and\_powerpoint.}}\sphinxbfcode{\sphinxupquote{plot\_wanted\_matrices}}}{\sphinxparam{\DUrole{n,n}{wafer\_id}}, \sphinxparam{\DUrole{n,n}{sessions}}, \sphinxparam{\DUrole{n,n}{structures}}, \sphinxparam{\DUrole{n,n}{types}}, \sphinxparam{\DUrole{n,n}{Temps}}, \sphinxparam{\DUrole{n,n}{Files}}, \sphinxparam{\DUrole{n,n}{coords}}}{}
\pysigstopsignatures
\sphinxAtStartPar
Used to plot only selected dies following selected filters. These plots won’t be registered but will be displayed for the user.
One plot is created per type of Measurement.
A single plot contain all session, all dies and all structures selected. A list of 15 colors is generated, so up to 15 differents lines can be differenciated.
You can add a color in this list if you want (Line 44 to 60) but don’t forget to change the number in line 109 (color\_index \% \textless{}this number\textgreater{})
\begin{quote}\begin{description}
\sphinxlineitem{Parameters}\begin{itemize}
\item {} 
\sphinxAtStartPar
\sphinxstyleliteralstrong{\sphinxupquote{wafer\_id}} \textendash{} ID of the wafer

\item {} 
\sphinxAtStartPar
\sphinxstyleliteralstrong{\sphinxupquote{sessions}} \textendash{} Selected sessions

\item {} 
\sphinxAtStartPar
\sphinxstyleliteralstrong{\sphinxupquote{structures}} \textendash{} Selected structures

\item {} 
\sphinxAtStartPar
\sphinxstyleliteralstrong{\sphinxupquote{types}} \textendash{} Selected types

\item {} 
\sphinxAtStartPar
\sphinxstyleliteralstrong{\sphinxupquote{Temps}} \textendash{} Selected temperatures

\item {} 
\sphinxAtStartPar
\sphinxstyleliteralstrong{\sphinxupquote{Files}} \textendash{} Selected files

\item {} 
\sphinxAtStartPar
\sphinxstyleliteralstrong{\sphinxupquote{coords}} \textendash{} Selected coordinates

\end{itemize}

\sphinxlineitem{Returns}
\sphinxAtStartPar
One plot per type of measurements, converted to base64

\end{description}\end{quote}

\end{fulllineitems}

\index{wanted\_ppt() (in module plot\_and\_powerpoint)@\spxentry{wanted\_ppt()}\spxextra{in module plot\_and\_powerpoint}}

\begin{fulllineitems}
\phantomsection\label{\detokenize{plot_and_powerpoint:plot_and_powerpoint.wanted_ppt}}
\pysigstartsignatures
\pysiglinewithargsret{\sphinxcode{\sphinxupquote{plot\_and\_powerpoint.}}\sphinxbfcode{\sphinxupquote{wanted\_ppt}}}{\sphinxparam{\DUrole{n,n}{wafer\_id}}, \sphinxparam{\DUrole{n,n}{sessions}}, \sphinxparam{\DUrole{n,n}{structures}}, \sphinxparam{\DUrole{n,n}{types}}, \sphinxparam{\DUrole{n,n}{Temps}}, \sphinxparam{\DUrole{n,n}{Files}}, \sphinxparam{\DUrole{n,n}{coords}}, \sphinxparam{\DUrole{n,n}{file\_name}}}{}
\pysigstopsignatures
\sphinxAtStartPar
Used to create a PowerPoint (registered in a folder named after the concerned wafer) with only selected dies following selected filters. These plots will be registered in a folder named ‘plots’.
One plot is created per type of Measurement.
A single plot contain all session, all dies and all structures selected. A list of 15 colors is generated, so up to 15 differents lines can be differenciated on the plot.
You can add a color in this list if you want (Line 160 to 176) but don’t forget to change the number in line 227 (color\_index \% \textless{}this number\textgreater{})
\begin{quote}\begin{description}
\sphinxlineitem{Parameters}\begin{itemize}
\item {} 
\sphinxAtStartPar
\sphinxstyleliteralstrong{\sphinxupquote{wafer\_id}} \textendash{} ID of the wafer

\item {} 
\sphinxAtStartPar
\sphinxstyleliteralstrong{\sphinxupquote{sessions}} \textendash{} Selected sessions

\item {} 
\sphinxAtStartPar
\sphinxstyleliteralstrong{\sphinxupquote{structures}} \textendash{} Selected structures

\item {} 
\sphinxAtStartPar
\sphinxstyleliteralstrong{\sphinxupquote{types}} \textendash{} Selected types

\item {} 
\sphinxAtStartPar
\sphinxstyleliteralstrong{\sphinxupquote{Temps}} \textendash{} Selected temperatures

\item {} 
\sphinxAtStartPar
\sphinxstyleliteralstrong{\sphinxupquote{Files}} \textendash{} Selected files

\item {} 
\sphinxAtStartPar
\sphinxstyleliteralstrong{\sphinxupquote{coords}} \textendash{} Selected coordinates

\item {} 
\sphinxAtStartPar
\sphinxstyleliteralstrong{\sphinxupquote{file\_name}} \textendash{} Name given to the file

\end{itemize}

\sphinxlineitem{Returns}
\sphinxAtStartPar
One plot per type of measurements, converted to base64

\end{description}\end{quote}

\end{fulllineitems}


\sphinxstepscope


\section{split\_data module}
\label{\detokenize{split_data:module-split_data}}\label{\detokenize{split_data:split-data-module}}\label{\detokenize{split_data::doc}}\index{module@\spxentry{module}!split\_data@\spxentry{split\_data}}\index{split\_data@\spxentry{split\_data}!module@\spxentry{module}}\index{C\_spliter() (in module split\_data)@\spxentry{C\_spliter()}\spxextra{in module split\_data}}

\begin{fulllineitems}
\phantomsection\label{\detokenize{split_data:split_data.C_spliter}}
\pysigstartsignatures
\pysiglinewithargsret{\sphinxcode{\sphinxupquote{split\_data.}}\sphinxbfcode{\sphinxupquote{C\_spliter}}}{\sphinxparam{\DUrole{n,n}{line}}}{}
\pysigstopsignatures
\sphinxAtStartPar
This function takes a line of datas from a .txt file containing C\sphinxhyphen{}V measurements in argument and returns a list with voltage in first position, RS in second position and CS in third position
\begin{quote}\begin{description}
\sphinxlineitem{Parameters}
\sphinxAtStartPar
\sphinxstyleliteralstrong{\sphinxupquote{line}} (\sphinxstyleliteralemphasis{\sphinxupquote{\textless{}str\textgreater{}}}) \textendash{} The line you want to extract information

\sphinxlineitem{Returns}
\sphinxAtStartPar
information needed

\sphinxlineitem{Return type}
\sphinxAtStartPar
list of str

\end{description}\end{quote}

\end{fulllineitems}

\index{converter\_split() (in module split\_data)@\spxentry{converter\_split()}\spxextra{in module split\_data}}

\begin{fulllineitems}
\phantomsection\label{\detokenize{split_data:split_data.converter_split}}
\pysigstartsignatures
\pysiglinewithargsret{\sphinxcode{\sphinxupquote{split\_data.}}\sphinxbfcode{\sphinxupquote{converter\_split}}}{\sphinxparam{\DUrole{n,n}{line}}}{}
\pysigstopsignatures
\sphinxAtStartPar
Used to handle lines when getting data from a tbl file
\begin{quote}\begin{description}
\sphinxlineitem{Parameters}
\sphinxAtStartPar
\sphinxstyleliteralstrong{\sphinxupquote{line}} (\sphinxstyleliteralemphasis{\sphinxupquote{\textless{}str\textgreater{}}}) \textendash{} Line to be handled

\sphinxlineitem{Returns}
\sphinxAtStartPar
Line handled

\end{description}\end{quote}

\end{fulllineitems}

\index{converter\_split\_session() (in module split\_data)@\spxentry{converter\_split\_session()}\spxextra{in module split\_data}}

\begin{fulllineitems}
\phantomsection\label{\detokenize{split_data:split_data.converter_split_session}}
\pysigstartsignatures
\pysiglinewithargsret{\sphinxcode{\sphinxupquote{split\_data.}}\sphinxbfcode{\sphinxupquote{converter\_split\_session}}}{\sphinxparam{\DUrole{n,n}{line}}}{}
\pysigstopsignatures
\sphinxAtStartPar
Used to get the name of the session when getting data from a tbl file.
\begin{quote}\begin{description}
\sphinxlineitem{Parameters}
\sphinxAtStartPar
\sphinxstyleliteralstrong{\sphinxupquote{line}} \textendash{} Line to be handled

\sphinxlineitem{Returns}
\sphinxAtStartPar
Line handled

\end{description}\end{quote}

\end{fulllineitems}

\index{dataSpliter() (in module split\_data)@\spxentry{dataSpliter()}\spxextra{in module split\_data}}

\begin{fulllineitems}
\phantomsection\label{\detokenize{split_data:split_data.dataSpliter}}
\pysigstartsignatures
\pysiglinewithargsret{\sphinxcode{\sphinxupquote{split\_data.}}\sphinxbfcode{\sphinxupquote{dataSpliter}}}{\sphinxparam{\DUrole{n,n}{line}}}{}
\pysigstopsignatures
\sphinxAtStartPar
This function takes a line of datas from a .txt file in argument and returns a list with voltage in first position and current in second position
\begin{quote}\begin{description}
\sphinxlineitem{Parameters}
\sphinxAtStartPar
\sphinxstyleliteralstrong{\sphinxupquote{line}} (\sphinxstyleliteralemphasis{\sphinxupquote{\textless{}str\textgreater{}}}) \textendash{} The line you want to extract the information

\sphinxlineitem{Returns}
\sphinxAtStartPar
the information needed

\sphinxlineitem{Return type}
\sphinxAtStartPar
list of str

\end{description}\end{quote}

\end{fulllineitems}

\index{spliter() (in module split\_data)@\spxentry{spliter()}\spxextra{in module split\_data}}

\begin{fulllineitems}
\phantomsection\label{\detokenize{split_data:split_data.spliter}}
\pysigstartsignatures
\pysiglinewithargsret{\sphinxcode{\sphinxupquote{split\_data.}}\sphinxbfcode{\sphinxupquote{spliter}}}{\sphinxparam{\DUrole{n,n}{line}}}{}
\pysigstopsignatures
\sphinxAtStartPar
This function takes a line of a header from a .txt file in argument and returns the relevant information in this line
\begin{quote}\begin{description}
\sphinxlineitem{Parameters}
\sphinxAtStartPar
\sphinxstyleliteralstrong{\sphinxupquote{line}} (\sphinxstyleliteralemphasis{\sphinxupquote{\textless{}str\textgreater{}}}) \textendash{} The line you want to extract information

\sphinxlineitem{Returns}
\sphinxAtStartPar
Information needed

\sphinxlineitem{Return type}
\sphinxAtStartPar
str

\end{description}\end{quote}

\end{fulllineitems}


\sphinxstepscope


\section{test\_DataExtraction module}
\label{\detokenize{test_DataExtraction:module-test_DataExtraction}}\label{\detokenize{test_DataExtraction:test-dataextraction-module}}\label{\detokenize{test_DataExtraction::doc}}\index{module@\spxentry{module}!test\_DataExtraction@\spxentry{test\_DataExtraction}}\index{test\_DataExtraction@\spxentry{test\_DataExtraction}!module@\spxentry{module}}\index{test\_filter() (in module test\_DataExtraction)@\spxentry{test\_filter()}\spxextra{in module test\_DataExtraction}}

\begin{fulllineitems}
\phantomsection\label{\detokenize{test_DataExtraction:test_DataExtraction.test_filter}}
\pysigstartsignatures
\pysiglinewithargsret{\sphinxcode{\sphinxupquote{test\_DataExtraction.}}\sphinxbfcode{\sphinxupquote{test\_filter}}}{}{}
\pysigstopsignatures
\end{fulllineitems}


\sphinxstepscope


\chapter{test module}
\label{\detokenize{test:test-module}}\label{\detokenize{test::doc}}

\chapter{Indices and tables}
\label{\detokenize{index:indices-and-tables}}\begin{itemize}
\item {} 
\sphinxAtStartPar
\DUrole{xref,std,std-ref}{genindex}

\item {} 
\sphinxAtStartPar
\DUrole{xref,std,std-ref}{modindex}

\item {} 
\sphinxAtStartPar
\DUrole{xref,std,std-ref}{search}

\end{itemize}


\renewcommand{\indexname}{Python Module Index}
\begin{sphinxtheindex}
\let\bigletter\sphinxstyleindexlettergroup
\bigletter{a}
\item\relax\sphinxstyleindexentry{app}\sphinxstyleindexpageref{app:\detokenize{module-app}}
\indexspace
\bigletter{c}
\item\relax\sphinxstyleindexentry{converter}\sphinxstyleindexpageref{converter:\detokenize{module-converter}}
\indexspace
\bigletter{e}
\item\relax\sphinxstyleindexentry{excel}\sphinxstyleindexpageref{excel:\detokenize{module-excel}}
\indexspace
\bigletter{f}
\item\relax\sphinxstyleindexentry{filter}\sphinxstyleindexpageref{filter:\detokenize{module-filter}}
\indexspace
\bigletter{g}
\item\relax\sphinxstyleindexentry{getter}\sphinxstyleindexpageref{getter:\detokenize{module-getter}}
\indexspace
\bigletter{n}
\item\relax\sphinxstyleindexentry{new\_manage\_DB}\sphinxstyleindexpageref{new_manage_DB:\detokenize{module-new_manage_DB}}
\item\relax\sphinxstyleindexentry{normal\_plots}\sphinxstyleindexpageref{normal_plots:\detokenize{module-normal_plots}}
\indexspace
\bigletter{p}
\item\relax\sphinxstyleindexentry{plot\_and\_powerpoint}\sphinxstyleindexpageref{plot_and_powerpoint:\detokenize{module-plot_and_powerpoint}}
\indexspace
\bigletter{s}
\item\relax\sphinxstyleindexentry{split\_data}\sphinxstyleindexpageref{split_data:\detokenize{module-split_data}}
\indexspace
\bigletter{t}
\item\relax\sphinxstyleindexentry{test\_DataExtraction}\sphinxstyleindexpageref{test_DataExtraction:\detokenize{module-test_DataExtraction}}
\indexspace
\bigletter{v}
\item\relax\sphinxstyleindexentry{VBD}\sphinxstyleindexpageref{VBD:\detokenize{module-VBD}}
\indexspace
\bigletter{w}
\item\relax\sphinxstyleindexentry{WaferMaps}\sphinxstyleindexpageref{WaferMaps:\detokenize{module-WaferMaps}}
\end{sphinxtheindex}

\renewcommand{\indexname}{Index}
\printindex
\end{document}